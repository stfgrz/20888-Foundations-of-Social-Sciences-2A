\vspace{-0.5cm}

    \subsection*{Historical Context}

        \begin{itemize}
          \item \textbf{Protestant Reformation (1517)}
            \begin{itemize}
              \item Initiated by Martin Luther, leading to significant religious and political upheaval in Europe.
              \item Contributed to the fragmentation of the Catholic Church and fueled inter-state and intra-state conflicts.
            \end{itemize}
        
          \item \textbf{European Religious Wars (16th--17th centuries and beyond)}
            \begin{itemize}
              \item Culmination of tensions between Protestants and Catholics, as well as between different Protestant factions.
              \item These conflicts ravaged many European regions, fostering an environment of mistrust, instability, and the need for strong governance.
            \end{itemize}
        
          \item \textbf{Scientific Revolution: Galileo (1564--1642), Descartes (1596--1650)}
            \begin{itemize}
              \item A shift toward empiricism and rational inquiry.
              \item Galileo’s emphasis on observation and mathematics, and Descartes’ focus on methodic doubt and the primacy of reason, challenged traditional Scholastic frameworks in philosophy and natural science.
            \end{itemize}
        
          \item \textbf{English Civil War (1642--1651)}
            \begin{itemize}
              \item Fought between Royalists (supporters of King Charles I) and Parliamentarians (supporters of parliamentary power).
              \item Thomas Hobbes lived through this period of turmoil. His experiences informed his conviction that strong central authority is necessary to prevent civil strife.
            \end{itemize}
        \end{itemize}

        These historical factors deeply influenced Hobbes’s thinking. The \textit{Protestant Reformation} and the ensuing religious wars showed him how divergent beliefs can destabilize entire societies, while the \textit{Scientific Revolution} shaped his mechanical, materialist approach to human cognition and social organization. The immediate political unrest of the \textit{English Civil War} offered him a firsthand example of the dangers of insufficiently centralized power.

\section{\textit{Leviathan} (1651)}

    Thomas Hobbes’s \textit{Leviathan} is one of the foundational texts of modern political philosophy. Published amidst the chaos of the English Civil War, it offers a rigorous defense of absolute sovereignty as a means to ensure peace and security. Before discussing the notion of the commonwealth, Hobbes establishes the philosophical basis of his approach by explaining his conception of sense, imagination, and the motions that drive human passions.

        \subsubsection{The Origin of Our Thoughts}

            \begin{quote}
                Concerning the thoughts of man, I will consider them first singly, and afterwards in train or dependence upon one another. Singly, they are every one a representation or appearance of some quality, or other accident of a body without us … The original of them all is that which we call SENSE … The rest are derived from that original.
            \end{quote}

            Hobbes begins by describing \textbf{thought} and \textbf{consciousness} as fundamentally connected to \textbf{sense experience}. For him, all cognition---every idea, concept, or thought---can be traced back to an external object pressing upon our sensory organs. This “pressing” or “motion” triggers internal changes that we register as sense impressions.

            \begin{remark}
                \begin{itemize}[leftmargin=*]
                    \item There is no content in the mind that did not first originate in sense (a direct challenge to the idea of innate knowledge).
                    \item Thought builds upon and rearranges these original sensory inputs over time.
                \end{itemize}
            \end{remark}

        \subsubsection{The Cause of Sense}

            \begin{quote}
            The cause of sense is the external body, or object, which presseth the organ proper to each sense … which pressure … continued inwards to the brain and heart, causeth there a resistance, or counter-pressure … which endeavour, because outward, seemeth to be some matter without.
            \end{quote}

            For Hobbes, \textbf{sense} arises through a mechanical process: external objects literally press upon our sense organs (directly for taste and touch, or indirectly for vision and hearing), causing a chain reaction of motions. Hobbes is adopting the nascent mechanical philosophy of his era, influenced by thinkers like Galileo, who emphasized that \textbf{all observable phenomena} (including perception) can be explained in terms of matter and motion.

        \subsubsection{Several Motions of the Matter}

            \begin{quote}
                All which qualities called sensible are in the object that causeth them but so many several motions of the matter … Neither in us that are pressed are they anything else but diverse motions … But their appearance to us is fancy, the same waking that dreaming.
            \end{quote}
            
            Here, Hobbes extends this mechanical explanation to clarify that what we perceive as “heat,” “cold,” “color,” or “sound” is actually the effect of \textit{various motions} in objects interacting with our organs. The “qualities” we perceive---like color---are not identical to anything literally colored existing outside us, but rather the result of our nervous system’s interpretation of motion.

In modern philosophy, this distinction is sometimes referred to as the difference between:
\begin{itemize}
  \item \textbf{Primary qualities} (e.g., extension, motion) which inhere in objects themselves.
  \item \textbf{Secondary qualities} (e.g., color, taste, sound) which are the mind’s interpretation of those motions.
\end{itemize}

\section*{The Object Is One Thing, the Image Is Another}

\begin{quote}
\textit{“... yet still the object is one thing, the image or fancy is another. So that sense in all cases is nothing else but original fancy caused … by the motion of external things upon our eyes, ears, and other organs ...”}
\end{quote}

This underlines Hobbes’s \textbf{representational} theory of perception: there is always a distinction between the external thing and the mental image or representation we have of it. He emphasizes that the process by which we see or hear something resembles the effect of physically pressing upon the relevant sense organ.

\section*{Chapter 1 -- \textit{Of Sense}}

\begin{remark}
\begin{itemize}
  \item The origin of thought is \textbf{sense}.
  \item The origin of sense is an \textbf{impression} (motion) from an external object.
  \item \textbf{Sense} is an \textbf{appearance} that does not coincide exactly with the object that caused it.
\end{itemize}
\end{remark}

\section*{Opposition to Scholasticism}

\begin{quote}
\textit{“But the philosophy schools, through all the universities of Christendom, grounded upon certain texts of Aristotle, teach another doctrine … they say, for the cause of vision, that the thing seen sendeth forth on every side a visible species … I say not this, as disapproving the use of universities: but … the frequency of insignificant speech is one.”}
\end{quote}

\subsubsection*{Hobbes vs. Aristotelian-Scholastic Philosophy}
Hobbes rejects the \textbf{Scholastic} notion of “species” or “forms” emanating from objects. According to the Aristotelian tradition, a “visible species” travels to our eyes, or an “audible species” reaches our ears. Hobbes sees these explanations as outdated and unnecessarily obscure, preferring instead a mechanistic, matter-in-motion account of perception.

\section*{Digression: Aristotle}

Hobbes draws heavily on certain parts of \textbf{Aristotle’s} thought (particularly regarding motion and potentiality), but discards other aspects (like species forms). Still, Aristotle’s concepts of \textit{dynamis} (potential) and \textit{energeia} (actualization) remain instructive:

\begin{remark}
There is not just one pressure or force, but two capacities or dispositions:
\begin{itemize}
  \item \textbf{Dynamis}: The potential or capacity (movement, tension)
  \item The meeting of two dispositions leads to \textbf{energeia} (the actualization)
\end{itemize}
\end{remark}

\subsubsection*{Aesthetics (Aisthesis)}
From the viewpoint of sensing (\textit{aisthesis}):
\begin{itemize}
  \item \textbf{Two dynamis}:
  \begin{enumerate}
    \item The table has the capacity or disposition to be seen.
    \item I have the capacity or disposition to see.
  \end{enumerate}
  \item \textbf{Energeia}:
    \begin{itemize}
      \item When these capacities meet, I \textit{actually} see the table, and the table \textit{is seen}.
    \end{itemize}
\end{itemize}

\subsubsection*{Technique (Techn\'e)}
From the viewpoint of \textit{techn\'e} (technique, or purposeful use):
\begin{itemize}
  \item \textbf{Two dynamis}:
  \begin{enumerate}
    \item The table has the capacity to be used (it is “ready” for use).
    \item I have the capacity to use it.
  \end{enumerate}
  \item \textbf{Energeia}:
    \begin{itemize}
      \item I \textit{use} the table, and the table \textit{is used} by me.
    \end{itemize}
\end{itemize}

In \textbf{Scholastic terms} (Aquinas), these are “potentia” and “actus.” In \textbf{Hobbesian} terms, power is simply \textit{motion} in matter. The important distinction is that Hobbes explains both sense and action via \textit{mechanical} cause and effect.

\section*{Chapter 2 -- \textit{Of Imagination}}

\begin{quote}
\textit{“That when a thing lies still … it will lie still for ever … is a truth that no man doubts of. But that when a thing is in motion, it will eternally be in motion, unless somewhat else stay it … is not so easily assented to.”}
\end{quote}

Hobbes explicitly references \textbf{Galileo’s} principle of inertia (bodies in motion continue in motion unless acted upon). He extends this to mental phenomena, arguing that \textbf{imagination} is simply the “decaying sense” left over after an original sensation has passed.

\section*{Chapter 6 -- \textit{Of the Interior Beginnings of Voluntary Motions, commonly called Passions}}

\subsubsection*{Vital and Animal Motion}

\begin{quote}
\textit{“There be in animals two sorts of motions peculiar to them: One called vital … the other is animal motion, otherwise called voluntary motion … that sense is motion … fancy is but the relics of the same motion … has been already said in the first and second chapters.”}
\end{quote}

\begin{itemize}
  \item \textbf{Vital motions}: Automatic processes (circulation, respiration, digestion) that do not require conscious thought.
  \item \textbf{Animal/Voluntary motions}: Actions we deliberately choose (walking, speaking) based on images or ideas in the mind.
\end{itemize}

\subsubsection*{Small Beginnings of Motion}

\begin{quote}
\textit{“These small beginnings of motion within the body of man, before they appear in walking … are commonly called endeavour.”}
\end{quote}

Hobbes introduces “\textbf{endeavour}” (\textit{conatus}): the minute, preconscious stirrings that set us on a particular path to act. Every deliberate act is preceded by this subtle motion in the mind and body.

\begin{itemize}
  \item \textbf{Desire} (or appetite) is endeavour \textbf{toward} something.
  \item \textbf{Aversion} is endeavour \textbf{away} from something.
\end{itemize}

\section*{Chapter 9 -- \textit{Of the Difference of Manners}}

\subsubsection*{Felicity Is Not Satisfaction}

\begin{quote}
\textit{“For there is no such finis ultimus (utmost aim) nor summum bonum (greatest good) … Felicity is a continual progress of the desire from one object to another, the attaining of the former being still but the way to the latter.”}
\end{quote}

Contrary to the ancient and medieval idea of a highest good or final end (e.g., \textit{eudaimonia} or \textit{beatitudo}), Hobbes argues that \textbf{human life} is characterized by \textbf{endless striving}. People are driven to seek power, comfort, and security continually; there is no final stopping point of perfect contentment.

\subsubsection*{A Restless Desire of Power After Power}

\begin{quote}
\textit{“... a perpetual and restless desire of power after power, that ceaseth only in death.”}
\end{quote}

This emphasis on \textbf{power} clarifies one of Hobbes’s most famous theses: humans are consistently compelled to increase their power---not necessarily out of boundless ambition, but often out of fear. Since one can never be sure that present power is sufficient for future security, individuals feel driven to acquire more.

\section*{Chapter 13 -- \textit{Of the Natural Condition of Mankind as concerning their Felicity and Misery}}

\subsubsection*{Diffidence of One Another}

\begin{quote}
\textit{“... if one plant, sow, build, or possess a convenient seat, others may probably be expected to come ... to dispossess and deprive him, not only of the fruit of his labour, but also of his life or liberty.”}
\end{quote}

In the \textbf{State of Nature}, Hobbes sees \textit{diffidence}---basic mistrust---of one another as inevitable. Since no central authority holds us in check, we are perpetually at risk of predation by others.

\subsubsection*{Anticipation}

\begin{quote}
\textit{“And from this diffidence of one another, there is no way for any man to secure himself so reasonable as anticipation ...”}
\end{quote}

Hobbes famously argues that a rational person in the state of nature will often strike first. Fear compels us to \textit{anticipate} aggression by others, leading us to preemptively attack. This dynamic feeds a cycle of insecurity and violence.

\subsubsection*{War Against All}

\begin{quote}
\textit{“... during the time men live without a common power to keep them all in awe, they are in that condition which is called war; and such a war as is of every man against every man.”}
\end{quote}

This condition is not limited to open combat but is any period during which there is a \textit{known disposition} to fight. Peace, by contrast, is the period during which mutual assurance exists under a power capable of enforcing it.

\section*{\textit{Homo Homini Lupus}}

This famous Latin phrase---“man is a wolf to man”---captures Hobbes’s view that humans, left to their own devices in the state of nature, revert to predatory behavior. Hobbes did not coin the phrase himself (it has roots in Plautus), but he popularized its application to early modern political thought.

\subsubsection*{No Place for Industry}

\begin{quote}
\textit{“In such condition there is no place for industry … and which is worst of all, continual fear, and danger of violent death; and the life of man, solitary, poor, nasty, brutish, and short.”}
\end{quote}

Hobbes explains that in a perpetual state of war and insecurity, \textbf{cultural and economic life} cannot flourish. There can be no advanced agriculture, no trade, no arts, and no lasting human cooperation because all fruits of labor remain exposed to theft or destruction.

\subsubsection*{No Pleasure in Keeping Company}

\begin{quote}
\textit{“Again, men have no pleasure … in keeping company where there is no power able to overawe them all.”}
\end{quote}

The entire purpose of coming together in \textbf{civil society}---establishing a commonwealth---is precisely to avoid this mutual fear. While humans are not \textit{naturally} social in Hobbes’s view (rejecting Aristotle’s \textit{zoon politikon} thesis), their recognition of danger eventually compels them to form society.

\begin{remark}
\begin{itemize}
  \item A \textbf{human being} is neither a \textit{zoon politikon} (naturally political animal) nor an \textit{animal sociale} (a naturally sociable being).
  \item \textbf{In the state of nature}, human beings are antisocial and in constant fear of one another.
  \item \textbf{Paradoxically}, it is this mutual fear that drives individuals to form a civil society under a powerful sovereign who enforces peace.
\end{itemize}
\end{remark}

\section*{Conclusion and What Might Be Missing}

\subsection*{Toward the Social Contract}
Although these notes end with Hobbes’s description of the “war of all against all,” a further crucial step is Hobbes’s argument about how rational individuals \textbf{escape} the state of nature. In chapters that follow (especially Chapters 14--17 of \textit{Leviathan}), Hobbes outlines how people, motivated by fear of violence, willingly give up certain natural rights to a sovereign power in a \textbf{social contract}. This sovereign---represented as the “Leviathan”---possesses enough strength to enforce peace and cooperation.

\subsection*{Mechanistic Worldview and Politics}
One of Hobbes’s \textit{unique contributions} is applying the mechanical worldview (inspired by Galileo and Descartes) to \textbf{political} and \textbf{ethical} questions. He explains both mental processes (imagination, desire, aversion) and social relations (the state of nature, conflict, the institution of sovereignty) in terms of bodies in motion, pressing and resisting one another.

\subsection*{Key Themes Often Discussed Alongside Hobbes}
\begin{itemize}
  \item \textbf{Epistemology and Science}: Hobbes’s insistence that sense comes from matter in motion places him among the early pioneers of modern empiricism.
  \item \textbf{Materialism}: Hobbes famously argued that even mental phenomena could be explained physically, challenging dualist philosophies.
  \item \textbf{Absolutism vs. Individual Rights}: Hobbes’s proposed solution---absolute sovereignty---was controversial, especially among those defending constitutional limitations.
  \item \textbf{Continuing Influence}: Hobbes’s ideas on fear, power, and contract theory laid groundwork for later Enlightenment thinkers like John Locke (who diverged on many points) and Jean-Jacques Rousseau.
\end{itemize}

\section*{Final Remarks}
These expanded notes cover both Hobbes’s \textbf{philosophical method} (rooted in the mechanical account of sense and imagination) and his \textbf{political vision} (the necessity of a common power to avoid the misery of war). The \textbf{historical context} underscores how the upheavals of the Reformation, religious strife, and civil war shaped Hobbes’s central claim: only a powerful, unified authority can tame the otherwise fearful and violent human condition.

If you plan to delve deeper into \textit{Leviathan}, consider exploring:
\begin{itemize}
  \item Hobbes’s \textbf{laws of nature} and how they underpin the social contract.
  \item His notion of \textbf{sovereignty}---the “mortal god”---and how it relates to individual liberty.
  \item His critique of alternative political theories, including rival contract theorists and the \textbf{Aristotelian-Scholastic} tradition he found so lacking.
\end{itemize}

All these elements together form one of the most influential and enduring works of modern political philosophy.
