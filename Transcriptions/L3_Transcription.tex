
    \subsection*{Historical Context}

        \begin{itemize}
            \item Protestant Reformation (1517)
            \item European Religious Wars (during the 16th and 17th and later)
            \item Scientific Revolution: Galileo (1564-1642), Descartes (1596-1650)
            \item English Civil War (Royalists vs Parliamentarians: 1642-1651)
        \end{itemize}

\section{\textit{Leviathan} (1651)}

        \subsubsection{The origin of our thoughts}

            \begin{quote}
                Concerning the thoughts of man, I will consider them first singly, and afterwards in train or dependence upon one another. Singly, they are every one a representation or appearance of some quality, or other accident of a body without us, which is commonly called an object. Which object worketh on the eyes, ears, and other parts of man’s body, and by diversity of working produceth diversity of appearances.  The original of them all is that which we call SENSE, (for there is no conception in a man’s mind which hath not at first, totally or by parts, been begotten upon the organs of sense). The rest are derived from that original. 
            \end{quote} 

        \subsubsection{The cause of sense}

            \begin{quote}
                The cause of sense is the external body, or object, which presseth the organ proper to each sense, either immediately, as in the taste and touch; or mediately, as in seeing, hearing, and smelling: which pressure, by the mediation of nerves and other strings and membranes of the body, continued inwards to the brain and heart, causeth there a resistance, or counter-pressure (anti-tupia), or endeavour of the heart to deliver itself: which endeavour, because outward, seemeth to be some matter without.
            \end{quote}

        \subsubsection{Several motions of the matter}

            \begin{quote}
                And this seeming, or fancy, is that which men call sense; and consisteth, as to the eye, in a light, or colour figured; to the ear, in a sound; to the nostril, in an odour; to the tongue and palate, in a savour; and to the rest of the body, in heat, cold, hardness, softness, and such other qualities as we discern by feeling. All which qualities called sensible are in the object that causeth them but so many several motions of the matter, by which it presseth our organs diversely. Neither in us that are pressed are they anything else but diverse motions (for motion produceth nothing but motion).
            \end{quote}

        \subsubsection{The object is one thing, the image is another}

            \begin{quote}
                But their appearance to us is fancy, the same waking that dreaming. And as pressing, rubbing, or striking the eye makes us fancy a light, and pressing the ear produceth a din; so do the bodies also we see, or hear, produce the same by their strong, though unobserved action…

                And though at some certain distance the real and very object seem invested with the fancy it begets in us; yet still the object is one thing, the image or fancy is another. So that sense in all cases is nothing else but original fancy caused (as I have said) by the pressure that is, by the motion of external things upon our eyes, ears, and other organs, thereunto ordained.
            \end{quote}

    \subsection[Of Sense]{Chapter 1 \\ \textit{Of Sense}}

        \begin{remark}
            \begin{itemize}
                \item The origin of thought is sense.
                \item The origin of sense is an impression of an external object.
                \item Sense is an appearance that does not coincide with the object that caused it.
            \end{itemize}
        \end{remark}

        \subsubsection{Opposition to Scholasticism}

            \begin{quote}
                But the philosophy schools, through all the universities of Christendom, grounded upon certain texts of Aristotle, teach another doctrine; and say, for the cause of vision, that the thing seen sendeth forth on every side a visible species, (in English) a visible show, apparition, or aspect, or a being seen; the receiving whereof into the eye is seeing. And for the cause of hearing, that the thing heard sendeth forth an audible species, that is, an audible aspect, or audible being seen; which, entering at the ear, maketh hearing. Nay, for the cause of understanding also, they say the thing understood sendeth forth an intelligible species, that is, an intelligible being seen; which, coming into the understanding, makes us understand.

                I say not this, as disapproving the use of universities: but because I am to speak hereafter of their office in a Commonwealth, I must let you see on all occasions by the way what things would be amended in them; amongst which the frequency of insignificant speech is one.
            \end{quote}

    \subsection*{Digression: Artistotle}

        \begin{remark}
            There is not just one pressure or external force, but two, two powers, or better, two capacities, two dispositions: \textit{dynamis} (dynamism, movement, tension)

            The meeting of two dispositions: \textit{energheia} (energy), the accomplishment
        \end{remark}

        \subsubsection{Aesthetics}

            From the point of view of the aistesis (aesthetics):
            
            \begin{itemize}
                \item \textbf{Two dynamis}: the disposition of the table to be seen and my disposition to see it
                \item \textbf{Energheia}: I see the table and the table is seen
            \end{itemize}

        \subsubsection{Technique}

            From the point of view of the techné (technique): 

            \begin{itemize}
                \item \textbf{Two dynamis}: the capacity of the table to be used (it is ready to be used), my capacity to use it
                \item \textbf{Energheia}: I use the table and the table is used by me
            \end{itemize}

            \begin{example}
                This table is in the accomplishment of its end (telos, entelecheia)
                
                The nature (physis) of the table: the capacity to support something (for example, the papers)
            \end{example}

            \begin{itemize}
                \item \textbf{Aquinas}: potentia (dynamis) / actus (energheia)
                \item \textbf{Hobbes}: power is action, movement of the matter
            \end{itemize}

            \begin{example}
                For example, the movements of an external objects that put in motion our bodies
            \end{example}

            Our resistance is a counter-pression, a counter movement (\textit{anti-tupia})

            Galileo: bodies that collide with each other

    \subsection[Of Imagination]{Chapter 2 \\ \textit{Of Imagination}}

        \subsubsection{Inertia}

            \begin{quote}
                That when a thing lies still, unless somewhat else stir it, it will lie still for ever, is a truth that no man doubts of. But that when a thing is in motion, it will eternally be in motion, unless somewhat else stay it, though the reason be the same (namely, that nothing can change itself), is not so easily assented to.
            \end{quote}

    \subsection[Of the Interior Beginnings of Voluntary Motions, commonly called Passions]{Chapter 6 \\ \textit{Of the Interior Beginnings of Voluntary Motions, commonly called Passions}}

        \subsubsection{Vital and animal motion}

            \begin{quote}
                There be in animals two sorts of motions peculiar to them: One called vital, begun in generation, and continued without interruption through their whole life; such as are the course of the blood, the pulse, the breathing, the concoction, nutrition, excretion, etc.; to which motions there needs no help of imagination: the other is animal motion, otherwise called voluntary motion; as to go, to speak, to move any of our limbs, in such manner as is first fancied in our minds. That sense is motion in the organs and interior parts of man’s body, caused by the action of the things we see, hear, etc., and that fancy is but the relics of the same motion, remaining after sense, has been already said in the first and second chapters.
            \end{quote} 

        \subsubsection{Small beginnings of motion}

            \begin{quote}
                And because going, speaking, and the like voluntary motions depend always upon a precedent thought of whither, which way, and what, it is evident that the imagination is the first internal beginning of all voluntary motion. And although unstudied men do not conceive any motion at all to be there, where the thing moved is invisible, or the space it is moved in is, for the shortness of it, insensible; yet that doth not hinder but that such motions are. For let a space be never so little, that which is moved over a greater space, whereof that little one is part, must first be moved over that. These small beginnings of motion within the body of man, before they appear in walking, speaking, striking, and other visible actions, are commonly called endeavour.
            \end{quote}

        \subsubsection{\textit{Conatus}}

            This endeavour (conatus), when it is toward something which causes it, is called appetite, or desire, the latter being the general name, and the other oftentimes restrained to signify the desire of food, namely hunger and thirst. And when the endeavour is from ward something, it is generally called aversion.

    \subsection[Of the Difference of Manners]{Chapter 9 \\ \textit{Of the Difference of Manners}}

        \subsubsection{Felicity is not satisfaction}

            \begin{quote}
                By manners, I mean not here decency of behaviour; as how one man should salute another, or how a man should wash his mouth, or pick his teeth before company, and such other points of the small morals; but those qualities of mankind that concern their living together in peace and unity. To which end we are to consider that the felicity of this life consisteth not in the repose of a mind satisfied. For there is no such finis ultimus (utmost aim) nor summum bonum (greatest good) as is spoken of in the books of the old moral philosophers. Nor can a man any more live whose desires are at an end than he whose senses and imaginations are at a stand. Felicity is a continual progress of the desire from one object to another, the attaining of the former being still but the way to the latter.
            \end{quote}

        \subsubsection{A restless desire of power after power}

            \begin{quote}
                So that in the first place, I put for a general inclination of all mankind a perpetual and restless desire of power after power, that ceaseth only in death… because he cannot assure the power and means to live well, which he hath present, without the acquisition of more.
                
                Desire of ease, and sensual delight, disposeth men to obey a common power.
            \end{quote}

    \subsection[Of the Natural Condition of Mankind as concerning their Felicity and Misery]{Chapter 13 \\ \textit{Of the Natural Condition of Mankind as concerning their Felicity and Misery}}

        \subsubsection{Diffidence of one another}

            \begin{quote}
                [...] if one plant, sow, build, or possess a convenient seat, others may probably be expected to come prepared with forces united to dispossess and deprive him, not only of the fruit of his labour, but also of his life or liberty.
            \end{quote}

        \subsubsection{Anticipation}

            \begin{quote}
                And from this diffidence of one another, there is no way for any man to secure himself so reasonable as anticipation; that is, by force, or wiles, to master the persons of all men he can so long till he see no other power great enough to endanger him: and this is no more than his own conservation requireth, and is generally allowed.
            \end{quote}

        \subsubsection{War against all}

            \begin{quote}
                Hereby it is manifest that during the time men live without a common power to keep them all in awe, they are in that condition which is called war; and such a war as is of every man against every man. For war consisteth not in battle only, or the act of fighting, but in a tract of time, wherein the will to contend by battle is sufficiently known: and therefore the notion of time is to be considered in the nature of war, as it is in the nature of weather. For as the nature of foul weather lieth not in a shower or two of rain, but in an inclination thereto of many days together: so the nature of war consisteth not in actual fighting, but in the known disposition thereto during all the time there is no assurance to the contrary. All other time is peace.
            \end{quote}

\section{\textit{Homo Homini Lupus}}

        \subsubsection{No place for industry}

            \begin{quote}
                In such condition there is no place for industry, because the fruit thereof is uncertain: and consequently no culture of the earth; no navigation, nor use of the commodities that may be imported by sea; no commodious building; no instruments of moving and removing such things as require much force; no knowledge of the face of the earth; no account of time; no arts; no letters; no society; and which is worst of all, continual fear, and danger of violent death; and the life of man, solitary, poor, nasty, brutish, and short.
            \end{quote}

        \subsubsection{No pleasure in keeping company}

            \begin{quote}
                Again, men have no pleasure (but on the contrary a great deal of grief) in keeping company where there is no power able to overawe them all.
            \end{quote}

    \begin{remark}
        Human being is neither a \textit{zoon politikon} nor an \textit{animal sociale}

        In the state of nature, human beings are antisocial

        But it is the antisocial principle (the fear of others’ fear) that brings them together in the civil state (under a common power)
    \end{remark}






