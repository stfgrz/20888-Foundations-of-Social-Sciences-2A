\section[\textit{The Fable of the Bees}]{\textit{The Fable of the Bees:} \\ \textit{or, Private Vices, Publick Benefits}}

    \subsection{Remark (B)}

    \subsection{Third Dialogue \\ Between Cleomenes and Horatio}

        \subsubsection{Origin of politeness?}

            \begin{quote}
                \textit{Horatio}: When I had the honour of your company at my house, you said, that nobody knew, when or where, nor in what King’s or Emperor’s reign the laws of honour were enacted; pray, can you inform me, when or which way, what we call good manners or politeness, came into the world? What moralist or politician was it, that could teach men to be proud of hiding their pride?
            \end{quote}

        \subsubsection{Self-love as self-preservation, but}

            \begin{quote}
                \textit{Cleomenes}: That self-love was given to all animals, at least, the most perfect, for self-preservation, is not disputed; but as no creature can love what it dislikes, it is necessary, moreover, that everyone should have a real liking to its own being, superior to what they have to any other. I am of opinion, begging pardon for the novelty, that if this liking was not always permanent, the love, which all creatures have for themselves, could not be so unalterable as we see it is.
            \end{quote}

            \begin{quote}
                \textit{Horatio}: What reason have you to suppose this liking, which creatures have for themselves, to be distinct from self-love; since the one plainly comprehends the other?
            \end{quote}

        \subsubsection{Self-liking, over-valuing oneself and vanity}

            \begin{quote}
                \textit{Cleomenes}: I will endeavour to explain myself better. I fancy, that, to increase the care in creatures to preserve themselves, nature has given them an instinct, by which every individual values itself above its real worth; this in us, I mean, in man, seems to be accompanied with a diffidence, arising from a consciousness, or at least an apprehension, that we do over-value ourselves. It is this that makes us so fond of the approbation, liking and assent of others; because they strengthen and confirm us in the good opinion we have of ourselves. The reasons why this self-liking give me leave to call it so, is not plainly to be seen in all animals that are of the same degree of perfection, are many.
            \end{quote}

        \subsubsection{No pleasure without it}

            \begin{quote}
                \textit{Cleomenes}: Whilst men are pleased, self-liking has every moment a considerable share, though unknown, in procuring the satisfaction they enjoy. It is so necessary to the well-being of those that have been used to indulge it; that they can taste no pleasure without it. It doubles our happiness in prosperity, and buoys us up against the frowns of adverse fortune. It is the mother of hopes, and the end as well as the foundation of our best wishes. It is the strongest armour against despair, and as long as we can like any ways our situation, either in regard to present circumstances, or the prospect before us, we take care of ourselves.
            \end{quote}

        \subsubsection{Self-love leading to suicide}

            \begin{quote}
                \textit{Cleomenes}: And no man can resolve upon suicide, whilst self-liking lasts; but as soon as that is over, all our hopes are extinct, and we can form no wishes but for the dissolution of our frame; till at last our being becomes so intolerable to us, that self-love prompts us to make an end of it and seek refuge in death.
                
                [...] how absurd soever a person’s reasoning may be, there is in all suicide a palpable intention of kindness to oneself.
            \end{quote}

        \subsubsection{Self-love as self-preservation [Cont'd]}

            \begin{quote}
                \textit{Cleomenes}: That self-love was given to all animals, at least, the most perfect, for self-preservation, is not disputed; but as no creature can love what it dislikes, it is necessary, moreover, that everyone should have a real liking to its own being, superior to what they have to any other. I am of opinion, begging pardon for the novelty, that if this liking was not always permanent, the love, which all creatures have for themselves, could not be so unalterable as we see it is.    
            \end{quote}

        \subsubsection{Spirit of superiority}

            \begin{quote}
                \textit{Cleomenes}: Whatever nature’s design was in bestowing this self-liking on creatures; and, whether it has been given to other animals besides ourselves or not, it is certain, that in our own species every individual person likes himself better than he does any other.
            \end{quote}

        \subsubsection{To wish to be another person}

            \begin{quote}
                \textit{Horatio}: It may be so, generally speaking; but that it is not universally true, I can assure you, from my own experience; for I have often wished myself to be Count Theodati, whom you knew at Rome.
            \end{quote}

        \subsubsection{To wish to be such another}

            \begin{quote}
                \textit{Cleomenes}: He was a very fine person indeed, and extremely well accomplished; and therefore you wished to be such another, which is all you could mean. Celia has a very handsome face, fine eyes, fine teeth; but she has red hair, and is ill made; therefore she wishes for Chloe’s hair and Bellinda’s shape; but she would still remain Celia.
            \end{quote}

        \subsubsection{Impossible to wish to be another person}

            \begin{quote}
                \textit{Horatio}: But I wished, that I might have been that person, that very Theodati.
            \end{quote}

            \begin{quote}
                \textit{Cleomenes}: That is impossible.
            \end{quote}

            \begin{quote}
                \textit{Horatio}: What, is it impossible to wish it!
            \end{quote}

        \subsubsection{\(\tau\)o self}

            \begin{quote}
                \textit{Cleomenes}: Yes, to wish it; unless you wished for annihilation at the same time. It is that self we wish well to; and therefore we cannot wish for any change in ourselves, but with a proviso, that that \(\tau\)o self, that part of us, that wishes, should still remain: for take away that consciousness you had of yourself, whilst you was wishing, and tell me pray, what part of you it is, that could be the better for the alteration you wished for?
            \end{quote}

        \subsubsection{The person wishing}

            \begin{quote}
                \textit{Horatio}: I believe you are in the right. No man can wish but to enjoy something, which no part of that same man could do, if he was entirely another.
            \end{quote}

            \begin{quote}
                \textit{Cleomenes}: That he itself, the person wishing, must be destroyed before the change could be entire.
            \end{quote}

    \subsection{Just before}

        \subsubsection{Vanity}

            \begin{quote}
                \textit{Cleomenes}: I believe, moreover, that many creatures shew this liking, when, for want of understanding them, we don’t perceive it. When a cat washes her face, and a dog licks himself clean, they adorn themselves as much as it is in their power. Man himself in a savage state, feeding on nuts and acorns, and destitute of all outward ornaments, would have infinitely less temptation, as well as opportunity, of shewing this liking of himself, than he has when civilized.
            \end{quote}

        \subsubsection{War before any agreement}

            \begin{quote}
                \textit{Cleomenes}: Yet if a hundred males of the first, all equally free, were together, within less than half an hour, this liking in question, though their bellies were full, would appear in the desire of superiority, that would be shewn among them; and the most vigorous, either in strength or understanding, or both, would be the first, that would display it. If, as supposed, they were all untaught, this would breed contention, and there would certainly be war before there could be any agreement among them…
            \end{quote}

    \subsection{Coming back}

        \subsubsection{Two equals}

            \begin{quote}
                
            \end{quote}
 
        \subsubsection{Both insufferable to each other}

            \begin{quote}
                
            \end{quote}

        \subsubsection{The unsocial sociability}

            \begin{quote}
                
            \end{quote}

        \subsubsection{Fruit of reflection?}

            \begin{quote}
                
            \end{quote}

        \subsubsection{Spontaneously}

            \begin{quote}
                
            \end{quote}

        \subsubsection{Stratagems without being aware of the causes}

            \begin{quote}
                
            \end{quote}

        \subsubsection{Even children}

            \begin{quote}
                
            \end{quote}

        \subsubsection{Two motions}

            \begin{quote}
                
            \end{quote}

        \subsubsection{Good manners without knowing their origins}

            \begin{quote}
                
            \end{quote}

        \subsubsection{Books come later}

            \begin{quote}
                
            \end{quote}

        \subsubsection{Practical knowledge}

            \begin{quote}
                
            \end{quote}

        \subsubsection{Guiding the vessel half a-sleep}

            \begin{quote}
                
            \end{quote}


            