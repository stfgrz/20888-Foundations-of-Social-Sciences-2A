\section{\textit{Summa Theologiae}}

    \begin{remark}
        Discusses Aristotle, especially \textit{Politics}, Book I, and \textit{Nicomachean Ethics}, Book V.
    \end{remark}

        \subsection{Question 77: On cheating in buying and selling}

            \subsubsection{Article 3. Whether the seller is bound to state the defects of the thing sold}

                \begin{quote}

                    Suppose, for example, a time of dearth and famine at Rhodes, with provisions at fabulous prices; and suppose that an honest man has imported a large cargo of grain from Alexandria and that to his certain knowledge also several other importers have set sail from Alexandria, and that on the voyage he has sighted their vessels laden with grain and bound for Rhodes; is he to report the fact to the Rhodians or is he to keep his own counsel and sell his own stock at the highest market price?

                    (Cicero, \textit{De Officiis}, Book 3.50)
                \end{quote}


                \begin{quote}
                    [...] it is not in accord with Nature that anyone should take advantage of his neighbour’s ignorance.

                    (Cicero, \textit{De Officiis}, Book 3.72)
                \end{quote}

                A defect in a thing lessens the value something seems to have in the present: but in the case cited, goods are expected to lessen in value at a point in the future, on arrival of other merchants of which buyers are unaware. Wherefore the seller who sells at the price as he finds it seems not to contravene justice through not exposing what is about to happen in the future. Were he to do so anyway, or to lower his price, then he would be exceedingly virtuous, however he does not seem to be held to do this because it were owed to justice.

                    \begin{quote}
                        ``\textit{Ad quartum dicendum quod vitium rei facit rem in praesenti esse minoris valoris quam videatur, sed in casu praemisso, in futurum res expectatur esse minoris valoris per superventum negotiatorum, qui ab ementibus ignoratur. Unde venditor qui vendit rem secundum pretium quod invenit, non videtur contra iustitiam facere si quod futurum est non exponat. Si tamen exponeret, vel de pretio subtraheret, abundantioris esset virtutis, quamvis ad hoc non videatur teneri ex iustitiae debito.}''
                    \end{quote}

                \begin{remark}
                    It is not a fraud but a change in circumstance (present/future).
                \end{remark}

                According to Aristotle, transactions ought to observe “equality of thing and thing”, says Aquinas, but there are exceptions.

            \subsubsection{Article 4. Whether, in trading, it is licit to sell a thing at a higher price than what was paid for it}

                \begin{quote}
                    A tradesman is one whose business consists in the exchange of things. According to the Philosopher\footnote{We are referring to Aristotle} (Polit. i, 3), exchange of things is twofold; one, natural as it were, and necessary, whereby one commodity is exchanged for another, or money taken in exchange for a commodity, in order to satisfy the needs of life. Such like trading, properly speaking, does not belong to tradesmen, but rather to housekeepers or civil servants who have to provide the household or the state with the necessaries of life. The other kind of exchange is either that of money for money, or of any commodity for money, not on account of the necessities of life, but for profit, and this kind of exchange, properly speaking, regards tradesmen, according to the Philosopher (Polit. i, 3). The former kind of exchange is commendable because it supplies a natural need: but the latter is justly deserving of blame, because, considered in itself, it satisfies the greed for gain, which knows no limit and tends to infinity.
                    
                    Hence trading, considered in itself, has a certain debasement attaching thereto, in so far as, by its very nature, it does not imply a virtuous or necessary end. Nevertheless gain which is the end of trading, though not implying, by its nature, anything virtuous or necessary, does not, in itself, connote anything sinful or contrary to virtue: wherefore nothing prevents gain from being directed to some necessary or even virtuous end, and thus trading becomes licit. Thus, for instance, a man may intend the moderate gain which he seeks to acquire by trading for the upkeep of his household, or for the assistance of the needy: or again, a man may take to trade for some public advantage, for instance, lest his country lack the necessaries of life, and seek gain, not as an end, but as payment for his labour (``\textit{stipendium laboris}'').

                    For if he sells at a higher price something that has changed for the better, he would seem to receive the reward of his labour.

                    For he may do this in a licit way, either because he has bettered the thing, or because the value of the thing has changed with the change of place or time, or on account of the danger he incurs in transferring the thing from one place to another, or again in having it carried by another. On this sense neither buying nor selling is unjust.
                \end{quote}

        \subsection{Question 61: The parts of Justice}

            The Philosopher says (Ethic. v, 3,4) that the mean in distributive justice is observed according to ``geometrical proportion'', whereas in commutative justice it follows “arithmetical proportion”.
                
            In distributive justice a person receives all the more of the common goods, according as he holds a more prominent position in the community. Hence in distributive justice the mean is observed, not according to equality between thing and thing, but according to proportion between things and persons: in such a way that even as one person surpasses another, so that which is given to one person surpasses that which is allotted to another.

            On the other hand, in commutations something is paid to an individual on account of something of his that has been received, as may be seen chiefly in selling and buying, where the notion of commutation is found primarily. Hence it is necessary to equalize thing with thing, so that the one person should pay back to the other just so much as he has become richer out of that which belonged to the other. The result of this will be equality according to the "arithmetical mean" which is gauged according to equal excess in quantity. Thus 5 is the mean between 6 and 4. Accordingly if, at the start, both persons have 5.

            If, however, we take for the matter of both kinds of justice the principal actions themselves, whereby we make use of persons, things, and works, there is then a difference of matter between them. For distributive justice directs distributions, while commutative justice directs commutations that can take place between two persons. Of these some are involuntary, some voluntary.

            \begin{remark}
                There must be proportionate reciprocation in exchange, but instead here Aquinas says that arithmetic mean is sufficient.

                He  is collapsing the third kind of justice into the second one, and this stays till today; it means a Judge can intervene and impose the just price.

                In a certain way, he's open to the commerce but he's introducing the fact that the authority can limit the commerce.
            \end{remark}

            They are involuntary when anyone uses another man's chattel, person, or work against his will, and this may be done secretly by fraud, or openly by violence. Voluntary commutations are when a man voluntarily transfers his chattel to another person. First when one man simply transfers his thing to another in exchange for another thing, as happens in selling and buying… On all these actions, whether voluntary or involuntary, the mean is taken in the same way according to the equality of repayment. Hence all these actions belong to the one same species of justice, namely commutative justice.

            Reciprocity [\textit{contrapassum}] denotes equal passion repaid for previous action; and the expression applies most properly to injurious passions and actions, whereby a man harms the person of his neighbor; for instance if a man strike, that he be struck back. This kind of just is laid down in the Law (Exodus 21:23-24): "He shall render life for life, eye for eye," etc. And since also to take away what belongs to another is to do an unjust thing, it follows that secondly reciprocity consists in this also, that whosoever causes loss to another, should suffer loss in his belongings, \textcolor{MidnightBlue}{reciprocity is transferred to voluntary commutations, where action and passion are on both sides, although voluntariness detracts from the nature of passion, as stated above.}

            In all these cases, however, repayment must be made on a basis of equality according to the requirements of commutative justice, namely that the meed of passion be equal to the action. Now there would not always be equality if passion were in the same species as the action. Because, in the first place, when a person injures the person of one who is greater, the action surpasses any passion of the same species that he might undergo, wherefore he that strikes a prince, is not only struck back, but is much more severely punished. On like manner when a man despoils another of his property against the latter's will, the action surpasses the passion if he be merely deprived of that thing, because the man who caused another's loss, himself would lose nothing, and so he is punished by making restitution several times over, because not only did he injure a private individual, but also the republic, the security of whose protection he has infringed.

            Nor again would there be equality of passion in voluntary commutations, were one always to exchange one's chattel for another man's, because it might happen that the other man's chattel is much greater than our own: so that it becomes necessary to equalize passion and action in commutations according to a certain proportionate commensuration, for which purpose money was invented. Hence reciprocity is in accordance with commutative justice: but there is no place for it in distributive justice, because in distributive justice we do not consider the equality between thing and thing or between passion and action (whence the expression 'contrapassum'), but according to proportion between things and persons, as stated above.

    \subsection*{Back to Question 77: On cheating in buying and selling}

        \subsubsection{Article 1. Whether it is licit to sell a thing for more than its value}

            It is altogether sinful to have recourse to deceit in order to sell a thing for more than its just price, because this is to deceive one's neighbour so as to injure him. But, apart from fraud, we may speak of buying and selling in two ways.

            First, as considered in themselves, and from this point of view, buying and selling seem to be established for the common advantage of both parties, one of whom requires that which belongs to the other, and vice versa, as the Philosopher states (Politics I, 1257, a, 6-9). Now whatever is established for the common advantage, should not be more of a burden to one party than to another, and consequently all contracts between them should observe equality of thing and thing. 
            Again, the quality of a thing that comes into human use is measured by the price given for it, for which purpose money was invented, as stated in Ethics V, 5. Therefore if either the price exceed the quantity of the thing’s value (``\textit{pretium excedat quantitatem valoris rei}''), or, conversely, the thing exceed the price, there is no longer the equality of justice: and consequently, to sell a thing for more than its value, or to buy it for less than its value, is in itself unjust and illicit.

            Secondly, we may speak of buying and selling, considered as accidentally tending to the advantage of one party, and to the disadvantage of the other: for instance, when a man has great need of a certain thing, while an other man will suffer if he be without it. On such a case the just price will depend not only on the thing sold, but on the loss which the sale brings on the seller. And thus it will be licit to sell a thing for more than it is value in itself (``\textit{licite poterit aliquid vendi plus quam valeat secundum se}''), though the price paid be not more than it is value to the owner.
            As stated above (I-II:96:2) human law is given to the people among whom there are many lacking virtue (``\textit{sunt multi a virtute deficientes}''), and it is not given to the virtuous alone. Hence human law was unable to forbid all that is contrary to virtue; and it suffices for it to prohibit whatever is destructive of human intercourse, while it treats other matters as though they were licit, not by approving of them, but by not punishing them.

            Accordingly, if without employing deceit the seller sells his goods overvaluing them (``\textit{rem suam supervendat}''), or the buyer obtain them for less, the law looks upon this as licit, and provides no punishment for so doing, unless the excess be too great, because then even human law demands restitution to be made, for instance if a man be deceived in regard to more than half the amount of the just price (``\textit{iusti pretii}'') of a thing [Cod. IV, xliv, De Rescind. Vend. 2,8.

            On the other hand, the Divine law leaves nothing unpunished that is contrary to virtue. Hence, according to the Divine law, it is reckoned illicit if the equality of justice be not observed in buying and selling: and he who has received more than he ought must make compensation to him that has suffered damage, if the damage be considerable (``\textit{si sit notabile damnum}'').

            I add this condition, because the just price of things (``\textit{iustum pretium rerum}'') is not precisely determined (``\textit{non est punctualiter determinatum}''), but consists in a kind of estimate (``\textit{in quadam aestimatione consistit}''), so that a slight addition or subtraction would not seem to destroy the equality of justice.

\section{Commentary on Aristotle's \textit{Nicomachean Ethics}}

    \begin{quote}
        971. Next […], he [Aristotle] explains in what matter and manner the statement is true that reciprocation is justice. He discusses this point from three aspects. First [III, A] he shows that there must be reciprocation in exchanges according to proportionality. Then […], he explains the form of this proportionality. Last [Lect. 9; C], at “Therefore all etc. ” (B. 1133 a 18), he shows how such a form can be observed. […] He says that in dealings of exchange it is true that justice is of such a nature that it includes reciprocation not according to equality but according to proportionality.
    \end{quote}

    \begin{quote}
        972. It seems this is contrary to what was said before (950), that in commutative justice the mean is taken not according to geometrical proportionality, which consists in an equality of proportion, but according to arithmetic proportionality, which consists in a quantitative equality. We must say that, in regard to commutative justice there should always be an equality of thing to thing, not, however, of action and passion, which implies corresponding requital. But in this, proportionality must be employed in order to bring about an equality of things because the work of one craftsman is of more value than the work of another, e.g., the building of a house than the production of a penknife. Hence, if the builder exchanged his work for the work of the cutler, there would not be equality of thing, given and taken, i.e., of house and penknife.
    \end{quote}

    \begin{quote}
        980. Next […], he shows how exchange takes place according to the preceding commensuration. Although a house is worth more than a sandal, nevertheless, a number of sandals are equal in value to one house or the food required for one man during a long period. In order then to have just exchange, as many sandals must be exchanged for one house or for the food required for one man as the builder or the farmer exceeds the shoemaker in his labour and costs. If this is not observed, there will be no exchange of things and men will not share their goods with one another. […]
    \end{quote}

    \begin{quote}
        983. Then […], he shows how just reciprocation takes place in exchanges according to the preceding commensuration. First [i, b, i] he explains his proposition; and then […] puts it in a diagram. He says first that the norm measuring all things by need according to nature and by currency according to human convention will then become reciprocation when everything will be equated in the way just mentioned. This is done in such a manner that as the farmer (whose work is raising food for men) excels the shoemaker (whose work is making sandals), in the same proportion the work of the shoemaker exceeds in number the work of the farmer, so that many sandals are exchanged for one bushel of wheat.
    \end{quote}

    \begin{quote}
        Thus, when exchange of things (``\textit{commutation rerum}'') takes place, the articles to be exchanged (``\textit{res commutandas}'') ought to be arranged in a proportional figure with diagonals, as was stated previously (957). If this was not done, one extreme would have both excesses (``\textit{superabundantias}'');

        if a farmer gave a bushel of wheat for a sandal, he would have an excess of labour in his product (``\textit{superabundantiam laboris in opere}'') and would have also an excess of presents (``\textit{superabundantiam doni}'') because he would be giving more than he would receive. But when all have what is theirs, they are in this way equal and do business with one another because the equality previously mentioned is possible for them.
    \end{quote}

            