\section[\textit{The Fable of the Bees}]{\textit{The Fable of the Bees:} \\ \textit{or, Private Vices, Public Benefits}}

    Bernard Mandeville’s \textit{The Fable of the Bees} (first published in 1714, with later expanded editions) famously argues that private “vices”—particularly those rooted in self-interest—can, under certain social arrangements, lead to public benefits such as economic prosperity, social order, and even politeness. The text below comes from a section of the work written in the form of dialogues between two characters, Cleomenes and Horatio, which further develops Mandeville’s insights into human nature. Much of the discussion centres on the idea of “self-liking” (distinct from mere self-preservation or “self-love”) and how it underlies vanity, social friction, and ultimately the development of good manners.

    \subsection{Remark (B)}

    \subsection[Third Dialogue Between Cleomenes and Horatio]{Third Dialogue \\ Between Cleomenes and Horatio}

        This dialogue explores how human passions—especially pride or “self-liking”—shape social behaviour. It shows Mandeville’s argument that what we call “politeness” or “good manners” ultimately grows out of these very impulses when people seek to coexist without continual conflict.

        \subsubsection{Origin of politeness?}

            \begin{quote}
                \textit{Horatio}: When I had the honour of your company at my house, you said, that nobody knew, when or where, nor in what King’s or Emperor’s reign the laws of honour were enacted; pray, can you inform me, when or which way, what we call good manners or politeness, came into the world? What moralist or politician was it, that could teach men to be proud of hiding their pride?
            \end{quote}

            Mandeville, through Horatio’s question, highlights the mysterious historical origins of what later becomes codified as “good manners.” The underlying point is that no single legislator, moralist, or philosopher consciously invented politeness. Instead, it arose gradually from deeper psychological and social needs—particularly the desire to mask and manage our inherent pride or self-liking in front of others.

        \subsubsection{Self-love as self-preservation, but}

            \begin{quote}
                \textit{Cleomenes}: That self-love was given to all animals, at least, the most perfect, for self-preservation, is not disputed; but as no creature can love what it dislikes, it is necessary, moreover, that everyone should have a real liking to its own being, superior to what they have to any other. I am of opinion, begging pardon for the novelty, that if this liking was not always permanent, the love, which all creatures have for themselves, could not be so unalterable as we see it is.
            \end{quote}

            \begin{quote}
                \textit{Horatio}: What reason have you to suppose this liking, which creatures have for themselves, to be distinct from self-love; since the one plainly comprehends the other?
            \end{quote}

            Here Mandeville distinguishes between self-preservation (the fundamental urge all creatures have to survive, which he calls “self-love”) and self-liking, a deeper psychological attachment to one’s own being that goes beyond mere survival.
            
            Self-liking implies a sort of positive, inflated self-assessment—often akin to pride or vanity. While self-love ensures we do not destroy ourselves, self-liking also motivates us to seek admiration or at least validation from others.

        \subsubsection{Self-liking, over-valuing oneself and vanity}

            \begin{quote}
                \textit{Cleomenes}: I will endeavour to explain myself better. I fancy, that, to increase the care in creatures to preserve themselves, nature has given them an instinct, by which every individual values itself above its real worth; this in us, I mean, in man, seems to be accompanied with a diffidence, arising from a consciousness, or at least an apprehension, that we do over-value ourselves. It is this that makes us so fond of the approbation, liking and assent of others; because they strengthen and confirm us in the good opinion we have of ourselves. The reasons why this self-liking give me leave to call it so, is not plainly to be seen in all animals that are of the same degree of perfection, are many.
            \end{quote}

            Mandeville claims that nature not only bestows self-preservation instincts but also a tendency to overvalue ourselves—an inborn pride or vanity. This overvaluation (i.e., vanity) coexists with a nagging sense that we might be exaggerating our own merits, which in turn drives us to seek social validation. The tension between “I am great” and “but perhaps I am not” leads us to crave praise.
            
            This dynamic underlies many social interactions and, for Mandeville, eventually sets the stage for “politeness,” because we must learn to navigate others’ equal sense of self-liking.

        \subsubsection{No pleasure without it}

            \begin{quote}
                \textit{Cleomenes}: Whilst men are pleased, self-liking has every moment a considerable share, though unknown, in procuring the satisfaction they enjoy. It is so necessary to the well-being of those that have been used to indulge it; that they can taste no pleasure without it. It doubles our happiness in prosperity, and buoys us up against the frowns of adverse fortune. It is the mother of hopes, and the end as well as the foundation of our best wishes. It is the strongest armour against despair, and as long as we can like any ways our situation, either in regard to present circumstances, or the prospect before us, we take care of ourselves.
            \end{quote}

            Mandeville here suggests that self-liking (that inward sense of overvaluing ourselves and our capacities) is a hidden ingredient in virtually all human enjoyment. When we feel good about ourselves, we interpret external events more positively and feel emotionally resilient—even when fortune turns against us.

        \subsubsection{Self-love leading to suicide}

            \begin{quote}
                \textit{Cleomenes}: And no man can resolve upon suicide, whilst self-liking lasts; but as soon as that is over, all our hopes are extinct, and we can form no wishes but for the dissolution of our frame; till at last our being becomes so intolerable to us, that self-love prompts us to make an end of it and seek refuge in death.
                
                [...] how absurd soever a person’s reasoning may be, there is in all suicide a palpable intention of kindness to oneself.
            \end{quote}

            At first glance, suicide seems contradictory to self-love. But Mandeville explains that the suicidal person’s self-liking has broken down, leaving only the minimal impetus of self-preservation. If living becomes more painful than dying, self-love paradoxically urges a person to end the pain.
            
            He is distinguishing psychological processes: once vanity, hope, or pride is lost, self-love “calculates” that death is less tormenting than continued existence. Thus, even suicide is—in a certain twisted way—still an expression of wanting to escape pain for one’s own sake.
        \subsubsection{Self-love as self-preservation [Cont'd]}

            \begin{quote}
                \textit{Cleomenes}: That self-love was given to all animals, at least, the most perfect, for self-preservation, is not disputed; but as no creature can love what it dislikes, it is necessary, moreover, that everyone should have a real liking to its own being, superior to what they have to any other. I am of opinion, begging pardon for the novelty, that if this liking was not always permanent, the love, which all creatures have for themselves, could not be so unalterable as we see it is.    
            \end{quote}

        \subsubsection{Spirit of superiority}

            \begin{quote}
                \textit{Cleomenes}: Whatever nature’s design was in bestowing this self-liking on creatures; and, whether it has been given to other animals besides ourselves or not, it is certain, that in our own species every individual person likes himself better than he does any other.
            \end{quote}

            Mandeville generalizes: the “spirit of superiority” is the inevitable outcome of self-liking. No matter how modest a person appears outwardly, each individual is prone to rank themselves higher in their own private estimation.

        \subsubsection{To wish to be another person}

            \begin{quote}
                \textit{Horatio}: It may be so, generally speaking; but that it is not universally true, I can assure you, from my own experience; for I have often wished myself to be Count Theodati, whom you knew at Rome.
            \end{quote}

        \subsubsection{To wish to be such another}

            \begin{quote}
                \textit{Cleomenes}: He was a very fine person indeed, and extremely well accomplished; and therefore you wished to be such another, which is all you could mean. Celia has a very handsome face, fine eyes, fine teeth; but she has red hair, and is ill made; therefore she wishes for Chloe’s hair and Bellinda’s shape; but she would still remain Celia.
            \end{quote}

            Cleomenes clarifies that to “wish to be another person” actually means to desire another’s qualities, not literally to annihilate one’s own consciousness. In Mandeville’s framework, we want to keep our core self—our “I”—while adding the attributes we admire in others.

        \subsubsection{Impossible to wish to be another person}

            \begin{quote}
                \textit{Horatio}: But I wished, that I might have been that person, that very Theodati.
            \end{quote}

            \begin{quote}
                \textit{Cleomenes}: That is impossible.
            \end{quote}

            \begin{quote}
                \textit{Horatio}: What, is it impossible to wish it!
            \end{quote}

        \subsubsection{\(\tau\)o self}

            \begin{quote}
                \textit{Cleomenes}: Yes, to wish it; unless you wished for annihilation at the same time. It is that self we wish well to; and therefore we cannot wish for any change in ourselves, but with a proviso, that that \(\tau\)o self, that part of us, that wishes, should still remain: for take away that consciousness you had of yourself, whilst you was wishing, and tell me pray, what part of you it is, that could be the better for the alteration you wished for?
            \end{quote}

        \subsubsection{The person wishing}

            \begin{quote}
                \textit{Horatio}: I believe you are in the right. No man can wish but to enjoy something, which no part of that same man could do, if he was entirely another.
            \end{quote}

            \begin{quote}
                \textit{Cleomenes}: That he itself, the person wishing, must be destroyed before the change could be entire.
            \end{quote}

            Here Mandeville offers a proto–philosophy-of-mind argument: a total swap of identities cannot accommodate the continuity of your own consciousness. Any real wish to be “someone else entirely” would involve destroying the self that is doing the wishing. In other words, the fundamental “I” that underpins our sense of self is non-transferable—one cannot both preserve that “I” and literally inhabit a different person’s being.

    \subsection{Just before}

        \subsubsection{Vanity}

            \begin{quote}
                \textit{Cleomenes}: I believe, moreover, that many creatures shew this liking, when, for want of understanding them, we don’t perceive it. When a cat washes her face, and a dog licks himself clean, they adorn themselves as much as it is in their power. Man himself in a savage state, feeding on nuts and acorns, and destitute of all outward ornaments, would have infinitely less temptation, as well as opportunity, of shewing this liking of himself, than he has when civilized.
            \end{quote}

            \begin{remark}
                Even animals show rudimentary self-liking by grooming themselves. Humans, when “uncivilized,” have fewer material means to display vanity, but in a “civilized” setting—replete with clothing, hairstyles, luxuries—opportunities for showing off and thereby fuelling vanity abound.
            \end{remark}

        \subsubsection{War before any agreement}

            \begin{quote}
                \textit{Cleomenes}: Yet if a hundred males of the first, all equally free, were together, within less than half an hour, this liking in question, though their bellies were full, would appear in the desire of superiority, that would be shewn among them; and the most vigorous, either in strength or understanding, or both, would be the first, that would display it. If, as supposed, they were all untaught, this would breed contention, and there would certainly be war before there could be any agreement among them…
            \end{quote}

            Mandeville suggests that if you gather a group of humans who have not been taught any social restraint, “self-liking” immediately pushes them to compete for dominance. This leads to inevitable conflict (“war”), unless social conventions and mutual agreements (laws, manners, hierarchies) eventually take shape.

    \subsection{Coming back}

            \begin{quote}
                \textit{Cleomenes}: That he itself, the person wishing, must be destroyed before the change could be entire.
            \end{quote}

            \begin{quote}
                \textit{Horatio}: But when shall we come to the origin of politeness?
            \end{quote}

        \subsubsection{Two equals}

            \begin{quote}
                \textit{Cleomenes}: We are at it now, and we need not look for it any further than in the self-liking, which I have demonstrated every individual man to be possessed of. Do but consider these two things; first, that from the nature of that passion it must follow, that all untaught men will ever be hateful to one another in conversation, where neither interest nor superiority are considered: for if of two equals one only values himself more by half, than he does the other; though that other should value the first equally with himself, they would both be dissatisfied, if their thoughts were known to each other.
            \end{quote}
 
        \subsubsection{Both insufferable to each other}

            \begin{quote}
                \textit{Cleomenes}: but if both valued themselves more by half, than they did each other, the difference between them would still be greater, and a declaration of their dentiments would render them both insufferable to each other; which among uncivilized men would happen every moment, because without a mixture of art and trouble, the outward symptoms of that passion are not to be stifled.
            \end{quote}

            Mandeville argues that mutual self-liking—where each person sees themselves as superior—makes social interaction unbearable if people do not suppress their open pride. Civility arises (in part) from the need to conceal or temper these inborn claims of superiority in order to avoid continual conflict.

        \subsubsection{The unsocial sociability}

            \begin{quote}
                \textit{Cleomenes}: The second thing I would have you consider, is, the effect which in all human probability this inconveniency, arising from self-liking, would have upon creatures, endued with a great share of understanding, that are fond of their ease to the last degree, and as industrious to procure it. These two things, I say, do but duly weigh, and you shall find, that the disturbance and uneasiness, that must be caused by self-liking, whatever struggling and unsuccessful trials to remedy them might precede, must necessarily produce at long run, what we call good manners and politeness.
            \end{quote}

            \begin{remark}
                Mandeville effectively anticipates what Kant would later call “unsocial sociability”: humans need to live together for comfort and productivity, yet our innate pride and vanity clash. 
                Through trial, error, and a shared desire to avoid tension, people gradually adopt “good manners” as a way of curbing open displays of pride.
            \end{remark}

        \subsubsection{Fruit of reflection?}

            \begin{quote}
                \textit{Horatio}: I understand you, I believe... the repeated experience of the uneasiness they received from such behaviour, would make some of them reflect on the cause of it; which, in tract of time, would make them find out, that their own barefaced pride must be as offensive to others, as that of others is to themselves.
            \end{quote}

        \subsubsection{Spontaneously}

            \begin{quote}
                \textit{Cleomenes}: What you say is certainly the philosophical reason of the alterations, that are made in the behaviour of men, by their being civilized; but all this is done without reflection, and men by degrees, and great length of time, fall as it were into these things spontaneously.
            \end{quote}

            Mandeville stresses that politeness is not merely a product of deliberate rational planning. It emerges “spontaneously” over long periods as people mimic, imitate, and adapt behaviours that lessen friction in social life.

        \subsubsection{Stratagems without being aware of the causes}

            \begin{quote}
                \textit{Cleomenes}: [...] it is incredible, how many useful cautions, shifts, and stratagems, they will learn to practise by experience and imitation, from conversing together; without being aware of the natural causes, that oblige them to act as they do. The passions within, that, unknown to themselves, govern their will and direct their behaviour.
            \end{quote}

            Good manners, in Mandeville’s analysis, are thus partly unconscious strategies. People refine their conduct by copying effective behaviours and avoiding those that cause rejection or ridicule, all the while driven by self-liking and a desire for social acceptance, even if they do not reflect on these deeper motivations.

        \subsubsection{Even children}

            \begin{quote}
                \textit{Cleomenes}: …without knowing anything of geometry or arithmetic, even children may learn to perform actions, that seem to bespeak great skill in mechanics, and a considerable depth of thought and ingenuity in the contrivance besides.
            \end{quote}

            \begin{quote}
                \textit{Horatio}: What actions are they, which you judge this from ?
            \end{quote}

            \begin{quote}
                \textit{Cleomenes}: The advantageous postures, which they will choose in resisting force, in pulling, pushing, or otherwise removing weight ; from their slight and dexterity in throwing stones, and other projectiles, and the stupendous cunning made use of in Leaping.
            \end{quote}

            By analogy, children “instinctively” acquire complex physical skills without understanding the mathematics or physics behind them. Likewise, adults learn manners without analysing the philosophy of pride, self-liking, or sociability.

        \subsubsection{Two motions}

            \begin{quote}
                \textit{Horatio}: What stupendous cunning, I pray?
            \end{quote}

            \begin{quote}
                \textit{Cleomenes}: When men would leap or jump a great way, you know, they take a run before they throw themselves off the ground. It is certain, that by this means they jump further, and with greater force than they could do otherwise; the reason likewise is very plain. The body partakes of, and is moved by, two motions; and the velocity, impressed upon it by leaping, must be added to so much, as it retained of the velocity it was put into by running. Whereas the body of a person who takes his leap, as he is standing still, has no other motion, than what is received from the muscular strength exerted in the act of leaping.
            \end{quote}

        \subsubsection{Good manners without knowing their origins}

            \begin{quote}
                \textit{Cleomenes}: See a thousand boys, as well as men, jump, and they will all make use of this stratagem; but you won’t find one of them, that does it knowingly for that reason. What I have said of this stratagem made use of in leaping, I desire you would apply to the doctrine of good manners, which is taught and practised by millions, who never thought on the origin of politeness, or so much as knew the real benefit it is of to society.
            \end{quote}

            Cleomenes explicitly draws the parallel between unconscious skill acquisition (like running before a leap) and the unreflective adoption of manners or polite behaviour.
            
            Most people learn “good manners” through imitation and habit rather than by theorizing about pride or “self-liking.” Yet the social result is the same: fewer conflicts and a smoother co-existence.

        \subsubsection{Books come later}

            \begin{quote}
                \textit{Cleomenes}: The chevalier Reneau wrote a book, in which he showed the mechanism of sailing, and accounts mathematically for everything that belongs to the working and steering of a ship. I am persuaded, that neither the first inventors of ships and sailing, or those, who have made improvements since in any part of them, ever dreamed of those reasons.
            \end{quote}

        \subsubsection{Practical knowledge}

            \begin{quote}
                \textit{Cleomenes}: The book I mentioned, among other curious things, demonstrates what angle the rudder must make with the keel, to render its influence upon the ship the most powerful. This has its merit; but a lad of fifteen, who has served a year of his time on board of a hoy, knows everything that is useful in this demonstration practically.
            \end{quote}

        \subsubsection{Guiding the vessel half a-sleep}

            \begin{quote}
                \textit{Cleomenes}: Seeing the poop always answering the motion of the helm, he only minds the latter, without making the least reflection on the rudder, till in a year or two more his knowledge in sailing, and capacity of steering his vessel become so habitual to him, that he guides her as he does his own body, by instinct, though he is half a-sleep, or thinking on quite another thing.
            \end{quote}

            These sections on sailing illustrate how practical knowledge long precedes (and often entirely bypasses) theoretical knowledge. People learn techniques that work—like adjusting the rudder or, by analogy, learning polite behaviours—even without understanding the underlying theoretical causes. Mandeville thus points out that societies can develop highly intricate customs, laws, and manners well before any philosopher formally describes why these customs arise or how they benefit society.

\newpage
\section*{Additional Context and Observations}

    \begin{remark}[Historical Reception]
        Mandeville’s work was highly controversial in the 18th century, accused of reducing morality to self-interest. Yet many later thinkers (including Hume, Smith, and Kant) grappled with or responded to his ideas in their own treatments of social order.
    \end{remark}    

    \begin{remark}[Comparison to Hobbes]
        Mandeville shares Hobbes’s view of natural human conflict but puts a greater emphasis on vanity and “self-liking,” whereas Hobbes primarily focuses on fear of violence and the pursuit of power.
    \end{remark}

    \begin{remark}[Connections to Modern Psychology]
        Contemporary discussions of self-esteem, self-deception, and social “performance” (e.g., Goffman’s Presentation of Self in Everyday Life) resonate strongly with Mandeville’s insights.
    \end{remark}

    \begin{remark}[Implications for Ethics and Economics]
        The Fable of the Bees arguably foreshadows ideas in classical economics (invisible hand–type arguments) and modern moral psychology, highlighting how self-interested impulses can structure large-scale social outcomes.
    \end{remark}

    \subsection*{Final Considerations}

        Mandeville’s core insight in these passages is that “self-liking” (or vanity) drives humans to seek comfort, validation, and admiration. In a state of nature or among “untaught men,” this same vanity leads to conflict. Yet over time, humans develop “good manners” and “politeness” as a spontaneous solution to managing and concealing their pride. Polite behavior, therefore, is not simply altruistic or the product of a pure moral law; it arises from the friction of competing self-liking and the collective realization (often without explicit reflection) that openly parading our pride becomes socially unbearable.

        Ultimately, Mandeville’s broader theme—central to The Fable of the Bees—is that private vices can yield public benefits. Our inclination to overvalue ourselves (a potential vice) motivates us to refine our social behaviour, thus giving rise to societal structures like manners, industry, and cooperation. In this sense, Mandeville challenges more idealistic philosophies that attribute civility to moral virtue alone. For him, it is a subtle dance of passions—especially vanity—that generates the order and refinement we call “civilization.”
            