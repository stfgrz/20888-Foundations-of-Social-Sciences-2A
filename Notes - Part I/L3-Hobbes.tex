\subsection*{Historical Context}

        \begin{itemize}
            \item \textbf{Protestant Reformation (1517)}
            \begin{itemize}
                \item Initiated by Martin Luther, leading to significant religious and political upheaval in Europe.
                \item Contributed to the fragmentation of the Catholic Church and fueled inter-state and intra-state conflicts.
            \end{itemize}
        
            \item \textbf{European Religious Wars (16th--17th centuries and beyond)}
            \begin{itemize}
                \item Culmination of tensions between Protestants and Catholics.
                \item These conflicts ravaged many European regions, fostering an environment of mistrust, instability, and the need for strong governance.
            \end{itemize}
        
            \item \textbf{Scientific Revolution: Galileo (1564--1642), Descartes (1596--1650)}
            \begin{itemize}
                \item A shift toward empiricism and rational inquiry.
                \item Galileo’s emphasis on observation and mathematics, and Descartes’ focus on methodic doubt and the primacy of reason, challenged traditional Scholastic frameworks in philosophy and natural science.
            \end{itemize}
        
            \item \textbf{English Civil War (1642--1651)}
            \begin{itemize}
                \item Fought between Royalists (supporters of King Charles I) and Parliamentarians (supporters of parliamentary power).
                \item Thomas Hobbes lived through this period of turmoil. His experiences informed his conviction that strong central authority is necessary to prevent chaos.
            \end{itemize}
        \end{itemize}
        
        These historical factors deeply influenced Hobbes’s thinking. The \textit{Protestant Reformation} and the ensuing religious wars showed him how divergent beliefs can destabilize entire societies, while the \textit{Scientific Revolution} shaped his mechanical, materialist approach to human cognition and social organization. The immediate political unrest of the \textit{English Civil War} offered him a firsthand example of the dangers of insufficiently centralized power.

\section{\textit{Leviathan} (1651)}

    Thomas Hobbes’s Leviathan is one of the foundational texts of modern political philosophy. Published amidst the chaos of the English Civil War, it offers a rigorous defense of absolute sovereignty as a means to ensure peace and security. Before discussing the notion of the commonwealth, Hobbes establishes the philosophical basis of his approach by explaining his conception of sense, imagination, and the motions that drive human passions.

        \subsubsection{The origin of our thoughts}

            \begin{quote}
                Concerning the thoughts of man, I will consider them first singly, and afterwards in train or dependence upon one another. Singly, they are every one a representation or appearance of some quality, or other accident of a body without us, which is commonly called an object. Which object worketh on the eyes, ears, and other parts of man’s body, and by diversity of working produceth diversity of appearances.  The original of them all is that which we call SENSE, (for there is no conception in a man’s mind which hath not at first, totally or by parts, been begotten upon the organs of sense). The rest are derived from that original. 
            \end{quote} 

            Hobbes begins by describing thought and consciousness as fundamentally connected to sense experience. For him, all cognition—every idea, concept, or thought—can be traced back to an external object pressing upon our sensory organs. This “pressing” or “motion” triggers internal changes that we register as sense impressions.

            \begin{remark}
                There is no content in the mind that did not first originate in sense (a direct challenge to the idea of innate knowledge). Thought builds upon and rearranges these original sensory inputs over time.
            \end{remark}

        \subsubsection{The cause of sense}

            \begin{quote}
                The cause of sense is the external body, or object, which presseth the organ proper to each sense, either immediately, as in the taste and touch; or mediately, as in seeing, hearing, and smelling: which pressure, by the mediation of nerves and other strings and membranes of the body, continued inwards to the brain and heart, causeth there a resistance, or counter-pressure (anti-tupia), or endeavour of the heart to deliver itself: which endeavour, because outward, seemeth to be some matter without.
            \end{quote}

            For Hobbes, sense arises through a mechanical process: external objects literally press upon our sense organs (directly for taste and touch, or indirectly for vision and hearing), causing a chain reaction of motions. Hobbes is adopting the nascent mechanical philosophy of his era, influenced by thinkers like Galileo, who emphasized that all observable phenomena (including perception) can be explained in terms of matter and motion.

\newpage
        \subsubsection{Several motions of the matter}

            \begin{quote}
                And this seeming, or fancy, is that which men call sense; and consisteth, as to the eye, in a light, or colour figured; to the ear, in a sound; to the nostril, in an odour; to the tongue and palate, in a savour; and to the rest of the body, in heat, cold, hardness, softness, and such other qualities as we discern by feeling. All which qualities called sensible are in the object that causeth them but so many several motions of the matter, by which it presseth our organs diversely. Neither in us that are pressed are they anything else but diverse motions (for motion produceth nothing but motion).
            \end{quote}

            Here, Hobbes extends this mechanical explanation to clarify that what we perceive as “heat,” “cold,” “color,” or “sound” is actually the effect of various motions in objects interacting with our organs. The “qualities” we perceive—like color—are not identical to anything literally colored existing outside us, but rather the result of our nervous system’s interpretation of motion.

            In modern philosophy, this distinction is sometimes referred to as the difference between:

            \begin{itemize}
                \item \textbf{Primary qualities} (e.g., extension, motion) which inhere in objects themselves.
                \item \textbf{Secondary qualities} (e.g., color, taste, sound) which are the mind’s interpretation of those motions.
            \end{itemize}

        \subsubsection{The object is one thing, the image is another}

            \begin{quote}
                But their appearance to us is fancy, the same waking that dreaming. And as pressing, rubbing, or striking the eye makes us fancy a light, and pressing the ear produceth a din; so do the bodies also we see, or hear, produce the same by their strong, though unobserved action…

                And though at some certain distance the real and very object seem invested with the fancy it begets in us; yet still the object is one thing, the image or fancy is another. So that sense in all cases is nothing else but original fancy caused (as I have said) by the pressure that is, by the motion of external things upon our eyes, ears, and other organs, thereunto ordained.
            \end{quote}

            This underlines Hobbes’s representational theory of perception: there is always a distinction between the external thing and the mental image or representation we have of it. He emphasizes that the process by which we see or hear something resembles the effect of physically pressing upon the relevant sense organ.

    \subsection[Of Sense]{Chapter 1 \\ \textit{Of Sense}}

        \begin{remark}
            \begin{itemize}
                \item The origin of thought is sense.
                \item The origin of sense is an impression of an external object.
                \item Sense is an appearance that does not coincide with the object that caused it.
            \end{itemize}
        \end{remark}

        \subsubsection{Opposition to Scholasticism}

            \begin{quote}
                But the philosophy schools, through all the universities of Christendom, grounded upon certain texts of Aristotle, teach another doctrine; and say, for the cause of vision, that the thing seen sendeth forth on every side a visible species, (in English) a visible show, apparition, or aspect, or a being seen; the receiving whereof into the eye is seeing. And for the cause of hearing, that the thing heard sendeth forth an audible species, that is, an audible aspect, or audible being seen; which, entering at the ear, maketh hearing. Nay, for the cause of understanding also, they say the thing understood sendeth forth an intelligible species, that is, an intelligible being seen; which, coming into the understanding, makes us understand.

                I say not this, as disapproving the use of universities: but because I am to speak hereafter of their office in a Commonwealth, I must let you see on all occasions by the way what things would be amended in them; amongst which the frequency of insignificant speech is one.
            \end{quote}

            Hobbes rejects the Scholastic notion of “species” or “forms” emanating from objects. According to the Aristotelian tradition, a “visible species” travels to our eyes, or an “audible species” reaches our ears. Hobbes sees these explanations as outdated and unnecessarily obscure, preferring instead a mechanistic, matter-in-motion account of perception.

    \subsection*{Digression: Artistotle}

        \begin{remark}
            There is not just one pressure or external force, but two, two powers, or better, two capacities, two dispositions: \textit{dynamis} (dynamism, movement, tension)

            The meeting of two dispositions: \textit{energheia} (energy), the accomplishment
        \end{remark}

        \subsubsection{Aesthetics (Aisthesis)}

            From the viewpoint of sensing (aisthesis):

            \begin{enumerate}
                \item Two dynamis:
                    \begin{itemize}
                        \item The table has the capacity or disposition to be seen.
                        \item I have the capacity or disposition to see.
                    \end{itemize}
                \item Energeia:
                    \begin{itemize}
                        \item When these capacities meet, I actually see the table, and the table is seen.
                    \end{itemize}
            \end{enumerate}

        \subsubsection{Technique}

            From the point of view of the techné (technique): 

            \begin{itemize}
                \item \textbf{Two dynamis}: the capacity of the table to be used (it is ready to be used), my capacity to use it
                \item \textbf{Energheia}: I use the table and the table is used by me
            \end{itemize}

            \begin{example}[Capacities and Use]
                Similarly, a table can be used, and a person has the capacity to use it. When these capacities come together, the object’s purpose (\textit{telos}) is realized—“the table is used.”
            \end{example}

            From Aristotle to Aquinas to Hobbes:
            
            \begin{itemize}
                \item \textbf{Aristotle \& Aquinas}: potentia (dynamis) vs. actus (energheia).
                \item \textbf{Hobbes}: Replaces these more “metaphysical” accounts with physical “motion of matter.”
            \end{itemize}

    \subsection[Of Imagination]{Chapter 2 \\ \textit{Of Imagination}}

        \subsubsection{Inertia}

            \begin{quote}
                That when a thing lies still, unless somewhat else stir it, it will lie still for ever, is a truth that no man doubts of. But that when a thing is in motion, it will eternally be in motion, unless somewhat else stay it, though the reason be the same (namely, that nothing can change itself), is not so easily assented to.
            \end{quote}

            Reflecting the post-Galilean world, Hobbes applies the principle of inertia to mental phenomena. Just as a physical object stays in motion unless stopped, an impression in the mind persists (as imagination) after the initial sensory input is removed.

    \subsection[Of the Interior Beginnings of Voluntary Motions, commonly called Passions]{Chapter 6 \\ \textit{Of the Interior Beginnings of Voluntary Motions, commonly called Passions}}

        \subsubsection{Vital and animal motion}

            \begin{quote}
                There be in animals two sorts of motions peculiar to them: One called vital, begun in generation, and continued without interruption through their whole life; such as are the course of the blood, the pulse, the breathing, the concoction, nutrition, excretion, etc.; to which motions there needs no help of imagination: the other is animal motion, otherwise called voluntary motion; as to go, to speak, to move any of our limbs, in such manner as is first fancied in our minds. That sense is motion in the organs and interior parts of man’s body, caused by the action of the things we see, hear, etc., and that fancy is but the relics of the same motion, remaining after sense, has been already said in the first and second chapters.
            \end{quote} 

            \begin{itemize}
                \item \textbf{Vital Motions}: Automatic bodily functions (heartbeat, digestion, breathing).
                \item \textbf{Animal (Voluntary) Motions}: Motions directed by conscious thought, e.g., walking, speaking, grasping an object.
            \end{itemize}

        \subsubsection{Small beginnings of motion}

            \begin{quote}
                And because going, speaking, and the like voluntary motions depend always upon a precedent thought of whither, which way, and what, it is evident that the imagination is the first internal beginning of all voluntary motion. And although unstudied men do not conceive any motion at all to be there, where the thing moved is invisible, or the space it is moved in is, for the shortness of it, insensible; yet that doth not hinder but that such motions are. For let a space be never so little, that which is moved over a greater space, whereof that little one is part, must first be moved over that. These small beginnings of motion within the body of man, before they appear in walking, speaking, striking, and other visible actions, are commonly called endeavour.
            \end{quote}

            \begin{itemize}
                \item \textbf{Precedent Thought}: Before any visible action (speaking, moving a limb), there is a “small,” internal motion in the mind—an endeavour or conatus.
                \item \textbf{Invisible Steps}: Hobbes highlights that even seemingly instantaneous decisions involve a micro-phase of mental motion that initiates the physical act.
            \end{itemize}

        \subsubsection{\textit{Conatus}}

            This endeavour (\textit{conatus}), when it is toward something which causes it, is called appetite, or desire, the latter being the general name, and the other oftentimes restrained to signify the desire of food, namely hunger and thirst. And when the endeavour is from ward something, it is generally called aversion.

            \begin{itemize}
                \item \textbf{Appetite} (Desire) vs. \textbf{Aversion}: If the mental motion is directed toward something, it is desire; if away, it is aversion.
                \item \textbf{Mechanistic Emotion}: Hobbes “mechanizes” the passions, framing even desire and fear as motions—pushing us either to approach or to retreat.
            \end{itemize}

    \subsection[Of the Difference of Manners]{Chapter 9 \\ \textit{Of the Difference of Manners}}

        \subsubsection{Felicity is not satisfaction}

            \begin{quote}
                By manners, I mean not here decency of behaviour; as how one man should salute another, or how a man should wash his mouth, or pick his teeth before company, and such other points of the small morals; but those qualities of mankind that concern their living together in peace and unity. To which end we are to consider that the felicity of this life consisteth not in the repose of a mind satisfied. For there is no such finis ultimus (utmost aim) nor summum bonum (greatest good) as is spoken of in the books of the old moral philosophers. Nor can a man any more live whose desires are at an end than he whose senses and imaginations are at a stand. Felicity is a continual progress of the desire from one object to another, the attaining of the former being still but the way to the latter.
            \end{quote}

            \begin{itemize}
                \item \textbf{No Ultimate End}: Rejects the ancient idea of a highest good (summum bonum). Humans never reach a point of permanent fulfillment; desire pushes us ceaselessly onward.
                \item \textbf{Restless Pursuit}: This perpetual seeking clarifies why humans strive for power and resources—there is no static state of contentment.
            \end{itemize}

        \subsubsection{A restless desire of power after power}

            \begin{quote}
                So that in the first place, I put for a general inclination of all mankind a perpetual and restless desire of power after power, that ceaseth only in death… because he cannot assure the power and means to live well, which he hath present, without the acquisition of more.
                
                Desire of ease, and sensual delight, disposeth men to obey a common power.
            \end{quote}

            \begin{itemize}
                \item \textbf{Security and Survival}: Hobbes traces our thirst for power to anxiety about our safety and well-being. We seek more power to ensure our current power is not undermined by others.
                \item \textbf{Roots of Conflict}: This universal drive can lead to competition and conflict. However, it also motivates people to accept a common power that can ensure they will not be harmed by rivals.
            \end{itemize}

    \subsection[Of the Natural Condition of Mankind as concerning their Felicity and Misery]{Chapter 13 \\ \textit{Of the Natural Condition of Mankind as concerning their Felicity and Misery}}

        \subsubsection{Diffidence of one another}

            \begin{quote}
                [...] if one plant, sow, build, or possess a convenient seat, others may probably be expected to come prepared with forces united to dispossess and deprive him, not only of the fruit of his labour, but also of his life or liberty.
            \end{quote}

            \begin{remark}[State of nature]
                Hobbes describes a scenario where there is no common power—no government, no laws, no shared enforcement mechanisms. In such a situation, trust is low, and the fear of losing one’s possessions (and life) is high.
            \end{remark}

        \subsubsection{Anticipation}

            \begin{quote}
                And from this diffidence of one another, there is no way for any man to secure himself so reasonable as anticipation; that is, by force, or wiles, to master the persons of all men he can so long till he see no other power great enough to endanger him: and this is no more than his own conservation requireth, and is generally allowed.
            \end{quote}

            \begin{proposition}[Preemptive Strikes]
                Feeling threatened, individuals may resort to attacking others first. This is deemed rational in the absence of a higher authority to guarantee safety.
            \end{proposition}

        \subsubsection{War against all}

            \begin{quote}
                Hereby it is manifest that during the time men live without a common power to keep them all in awe, they are in that condition which is called war; and such a war as is of every man against every man. For war consisteth not in battle only, or the act of fighting, but in a tract of time, wherein the will to contend by battle is sufficiently known: and therefore the notion of time is to be considered in the nature of war, as it is in the nature of weather. For as the nature of foul weather lieth not in a shower or two of rain, but in an inclination thereto of many days together: so the nature of war consisteth not in actual fighting, but in the known disposition thereto during all the time there is no assurance to the contrary. All other time is peace.
            \end{quote}

            \begin{itemize}
                \item \textbf{Definition of War}: War is not just combat but the known disposition to fight—an enduring state of suspicion and readiness to clash.
                \item \textbf{Time and Fear}: Just as “foul weather” involves an ongoing threat of storms, so “war” is a persistent threat of violence when no one can be sure of security.
            \end{itemize}

\section{\textit{Homo Homini Lupus}}

        \subsubsection{No place for industry}

            \begin{quote}
                In such condition there is no place for industry, because the fruit thereof is uncertain: and consequently no culture of the earth; no navigation, nor use of the commodities that may be imported by sea; no commodious building; no instruments of moving and removing such things as require much force; no knowledge of the face of the earth; no account of time; no arts; no letters; no society; and which is worst of all, continual fear, and danger of violent death; and the life of man, solitary, poor, nasty, brutish, and short.
            \end{quote}

            \begin{remark}[Consequences of Anarchy]
                 In this Hobbesian “state of nature,” the lack of security stifles all productive or cooperative pursuits—farming, trade, the arts, and science cannot flourish under constant fear of attack.
            \end{remark}

        \subsubsection{No pleasure in keeping company}

            \begin{quote}
                Again, men have no pleasure (but on the contrary a great deal of grief) in keeping company where there is no power able to overawe them all.
            \end{quote}

            \begin{itemize}
                \item \textbf{Human Beings as Antisocial by Nature}: Here, Hobbes rejects the Aristotelian idea that “man is by nature a political animal” (zoon politikon). Instead, absent an overawing power, humans are driven apart by mutual distrust.
                \item \textbf{Civil State as a Solution}: Ironically, it is precisely our antisocial self-interest—the fear of others’ aggression—that compels us to seek peace under a common authority (the Leviathan).
            \end{itemize}

    \begin{remark}
        Human being is neither a \textit{zoon politikon} nor an \textit{animal sociale}

        In the state of nature, human beings are antisocial

        But it is the antisocial principle (the fear of others’ fear) that brings them together in the civil state (under a common power)
    \end{remark}

\section*{Additional Context and Observations}

    \begin{remark}[Hobbes’s Biographical Context]
        He was tutored in classical languages, worked as a secretary and companion to noble families, and mingled with leading intellectuals in France and England. This shaped his blending of classical scholarship with cutting-edge scientific theory.
    \end{remark}

    \begin{remark}[Religious Overtones]
        Despite championing a materialist account, Hobbes was wary of being branded irreligious. He devotes substantial portions of Leviathan to scriptural interpretation, aiming to reconcile his political conclusions with a Christian framework.
    \end{remark}

    \begin{remark}[Comparison with Other Social Contract Theorists]
        Later writers (Locke, Rousseau, Kant) adopt the idea of a social contract but often reject Hobbes’s pessimistic anthropology. They propose that certain natural rights (e.g., property, liberty) must remain inalienable.
    \end{remark}

    \begin{remark}[Hobbes and Geometry]
        Hobbes fancied himself a geometer and believed that the clarity of geometric demonstration could be applied to moral and political philosophy. While he was not always successful in advanced geometry, this aspiration shaped his methodical presentation.
    \end{remark}

    \subsection*{Final Considerations}

        Hobbes’s Leviathan culminates in a social contract theory: individuals in the state of nature, driven by fear and rational self-interest, agree to yield their individual rights to a sovereign power in exchange for peace and security. This sovereign (whether a monarch or an assembly) must be all-powerful, otherwise it cannot enforce the peace effectively. Hobbes’s choice of the biblical sea monster “Leviathan” as a metaphor underscores the intimidating might required to compel obedience and protect subjects from each other.

        \begin{itemize}
            \item \textbf{Influence on Modern Political Thought}: Hobbes’s work laid a foundation for later contract theorists like John Locke and Jean-Jacques Rousseau, though they often took a more optimistic view of human nature.
            \item \textbf{Mechanical Materialism}: Hobbes extended the logic of the Scientific Revolution to explain both human cognition and social organization in terms of motion, pressure, and mechanistic causation.
            \item \textbf{Critiques and Legacy}: Critics argue Hobbes undervalues moral, spiritual, or civic virtues beyond self-preservation. Nonetheless, Leviathan remains indispensable for understanding the origins of modern concepts of sovereignty, the state, and the rule of law.
        \end{itemize}






