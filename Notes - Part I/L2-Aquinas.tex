\section{\textit{Summa Theologiae}}

    \begin{remark}
        Thomas Aquinas’s discussion of \textit{just price} and related issues in buying and selling draws heavily on Aristotle’s works, most notably \textit{Politics} (Book I) and \textit{Nicomachean Ethics} (Book V). It also engages with classical examples provided by Cicero (e.g., \textit{De Officiis}) and relies on Roman law regarding contracts and fairness (\textit{Codex Iustinianus}).
    \end{remark}

    \subsection{Question 77: On cheating in buying and selling}

        \subsubsection{Article 3. Whether the seller is bound to state the defects of the thing sold}

            \begin{quote}
                Suppose, for example, a time of dearth and famine at Rhodes, with provisions at fabulous prices; and suppose that an honest man has imported a large cargo of grain from Alexandria and that to his certain knowledge also several other importers have set sail from Alexandria, and that on the voyage he has sighted their vessels laden with grain and bound for Rhodes; is he to report the fact to the Rhodians or is he to keep his own counsel and sell his own stock at the highest market price?
            
                (Cicero, \textit{De Officiis}, Book 3.50)
            \end{quote}
    
            \begin{quote}
                [...] it is not in accord with Nature that anyone should take advantage of his neighbour’s ignorance.
            
                (Cicero, \textit{De Officiis}, Book 3.72)
            \end{quote}
    
            Aquinas distinguishes between hiding an actual, present defect in goods versus simply withholding information about future events. A defect in a thing lessens its value \emph{in the present}, while in the Rhodes example the coming arrival of additional grain will lessen the price \emph{in the future}. Since future circumstances are by definition uncertain to the buyer, Aquinas holds that the seller who sells at the current market price does not commit an injustice merely by not revealing the future influx of grain. He does acknowledge that revealing such information or voluntarily lowering the price would be more virtuous, but it is not strictly required by justice.
    
            \begin{quote}
                \textit{Ad quartum dicendum quod vitium rei facit rem in praesenti esse minoris valoris quam videatur, sed in casu praemisso, in futurum res expectatur esse minoris valoris per superventum negotiatorum, qui ab ementibus ignoratur. Unde venditor qui vendit rem secundum pretium quod invenit, non videtur contra iustitiam facere si quod futurum est non exponat. Si tamen exponeret, vel de pretio subtraheret, abundantioris esset virtutis, quamvis ad hoc non videatur teneri ex iustitiae debito.}
            \end{quote}
    
            \begin{remark}
            Aquinas concludes that such a situation is \emph{not} a fraud in the strict sense but rather a change in circumstance (present vs.\ future). In other words, if the item truly has a defect in the here and now, the seller is obliged to disclose it; if the expected lowering of the price is only in the future, one is not strictly obligated to mention it.
            \end{remark}
    
            Aquinas also draws on Aristotle’s principle that transactions \emph{ought} to observe an “equality of thing and thing.” Yet, in practical realities, exceptions may occur without constituting injustice.

        \subsubsection{Article 4. Whether, in trading, it is licit to sell a thing at a higher price than what was paid for it}

            \begin{quote}
                A tradesman is one whose business consists in the exchange of things. According to the Philosopher (Polit. i, 3), exchange of things is twofold: one, natural as it were, and necessary, whereby one commodity is exchanged for another, or money taken in exchange for a commodity, in order to satisfy the needs of life. Such like trading, properly speaking, does not belong to tradesmen, but rather to housekeepers or civil servants who have to provide the household or the state with the necessaries of life. 
            
                The other kind of exchange is either that of money for money, or of any commodity for money, not on account of the necessities of life, but for profit, and this kind of exchange, properly speaking, regards tradesmen, according to the Philosopher (Polit. i, 3). The former kind of exchange is commendable because it supplies a natural need: but the latter is justly deserving of blame, because, considered in itself, it satisfies the greed for gain, which knows no limit and tends to infinity.
            \end{quote}

            Aquinas remarks that trading for profit \emph{by itself} does not necessarily entail sinfulness, provided the intention is just. It becomes permissible if one’s ultimate goal is, for example, the upkeep of one’s household or the relief of the poor, rather than an endless accumulation of wealth. In such cases, a markup can be seen as “\textit{stipendium laboris},” i.e., compensation for labor, transport, or risk.

            \begin{quote}
                Hence trading, considered in itself, has a certain debasement attaching thereto, in so far as, by its very nature, it does not imply a virtuous or necessary end. Nevertheless gain which is the end of trading, though not implying, by its nature, anything virtuous or necessary, does not, in itself, connote anything sinful or contrary to virtue: wherefore nothing prevents gain from being directed to some necessary or even virtuous end, and thus trading becomes licit. [...] For if he sells at a higher price something that has changed for the better, he would seem to receive the reward of his labour.
            \end{quote}

            The key point is that if the item truly \emph{increases in value}—either because of the labor expended in improving it, or because of the costs (danger, transportation, etc.) involved—charging a higher price is legitimate and does not violate justice.

    \subsection{Question 61: The parts of Justice}

        Aquinas notes that Aristotle, in \textit{Nicomachean Ethics} Book V, distinguishes two basic forms of justice:
        \begin{enumerate}
            \item \textbf{Distributive justice}, whose mean is determined by \emph{geometrical proportion}. Those who are more “worthy” (e.g., possessing higher rank, ability, or contribution to the common good) receive proportionally more from the common stock.
            \item \textbf{Commutative (or Corrective) justice}, whose mean is determined by \emph{arithmetical proportion}. This applies to voluntary exchanges (like buying and selling) or involuntary exchanges (like theft or violence). Here we do not weigh personal status but rather strictly ensure that each party receives or restores an amount equal to what was lost or gained.
        \end{enumerate}

        \begin{quote}
            The Philosopher says (Ethic. v, 3,4) that the mean in distributive justice is observed according to ``geometrical proportion,'' whereas in commutative justice it follows “arithmetical proportion.” [...]
            
            On the other hand, in commutations something is paid to an individual on account of something of his that has been received [...] Hence it is necessary to equalize thing with thing, so that the one person should pay back to the other just so much as he has become richer out of that which belonged to the other. [...]
        \end{quote}

        \begin{remark}
            Aquinas \emph{collapses} many possible distinctions of exchange (including punitive “contrapassum”) into commutative justice. This allows a judge or an authority, if needed, to impose restitution or compensation and thereby restore equality.
        \end{remark}

        He also explains the notion of \emph{reciprocity} (\textit{contrapassum}), i.e.\ returning harm for harm, or restitution for what was taken:
        
        \begin{quote}
            Reciprocity [\textit{contrapassum}] denotes equal passion repaid for previous action; and the expression applies most properly to injurious passions and actions, whereby a man harms the person of his neighbor; for instance if a man strike, that he be struck back. [...] And since also to take away what belongs to another is to do an unjust thing, it follows that secondly reciprocity consists in this also, that whosoever causes loss to another, should suffer loss in his belongings, [...]. 
        \end{quote}
        
        Yet perfect equality is not always straightforward: striking a mere citizen versus striking a prince, for example, might receive different punishments to maintain the proportional or corrective “mean.” Likewise, theft or fraud in property may be punished by returning more than the stolen value, taking into account the harm to the wider community.

    \subsection*{Back to Question 77: On cheating in buying and selling}

        \subsubsection{Article 1. Whether it is licit to sell a thing for more than its value}

            Aquinas clarifies that employing deceit is always sinful. But absent deceit, there are two main considerations:
            
            \begin{enumerate}
                \item \emph{Buying and selling considered in themselves}. In principle, exchange is for the mutual advantage of both parties. This requires an equality of “thing and thing” and the avoidance of exploitation. Money was introduced exactly to measure this equivalence. If the \emph{price} unreasonably exceeds the \emph{thing’s value} (or vice versa), it violates commutative justice.
            
                \item \emph{Buying and selling considered in relation to special circumstances}. For instance, if the seller will incur a particular \emph{loss} or bears certain costs or risks, the “just price” can rightly be higher. Thus Aquinas allows selling at a higher price if that higher price genuinely reflects additional expenses, risks, or the utility the buyer personally attributes to the good.
            \end{enumerate}
            
            He also references how different legal or moral systems might respond. \emph{Human law} may tolerate certain lesser injustices and only punish more egregious ones (e.g., cases involving an overcharge greater than half the thing’s value, referencing Roman law on \textit{laesio enormis}). \emph{Divine law}, on the other hand, leaves nothing unpunished that is contrary to virtue. 

            \begin{quote}
                Accordingly, if without employing deceit the seller sells his goods overvaluing them (\textit{rem suam supervendat}), or the buyer obtain them for less, the law looks upon this as licit, and provides no punishment for so doing, unless the excess be too great, because then even human law demands restitution to be made [...]. 
                
                On the other hand, the Divine law leaves nothing unpunished that is contrary to virtue. [...] I add this condition, because the just price of things (\textit{iustum pretium rerum}) is not precisely determined (\textit{non est punctualiter determinatum}), but consists in a kind of estimate (\textit{in quadam aestimatione consistit}), so that a slight addition or subtraction would not seem to destroy the equality of justice.
            \end{quote}
            
            In other words, prices often involve an \emph{estimate} (because the “exact” value of a thing can be hard to establish). So minor deviations need not be considered injustice, but significant deviations may require restitution.

\section{Commentary on Aristotle's \textit{Nicomachean Ethics}}

    Aquinas’s \textit{Commentary on the Nicomachean Ethics} further discusses how just exchange depends on proportional equivalences:
    
    \begin{quote}
        971. Next […], he [Aristotle] explains in what matter and manner the statement is true that reciprocation is justice. He discusses this point from three aspects. First [III, A] he shows that there must be reciprocation in exchanges according to proportionality. Then […], he explains the form of this proportionality. Last [Lect. 9; C], at “Therefore all etc.” (B. 1133 a 18), he shows how such a form can be observed. […] He says that in dealings of exchange it is true that justice is of such a nature that it includes reciprocation not according to equality but according to proportionality.
    \end{quote}
    
    \begin{quote}
        972. It seems this is contrary to what was said before (950), that in commutative justice the mean is taken not according to geometrical proportionality, which consists in an equality of proportion, but according to arithmetic proportionality, which consists in a quantitative equality. We must say that, in regard to commutative justice there should always be an equality of thing to thing, not, however, of action and passion, which implies corresponding requital. But in this, proportionality must be employed in order to bring about an equality of things because the work of one craftsman is of more value than the work of another, e.g., the building of a house than the production of a penknife. Hence, if the builder exchanged his work for the work of the cutler, there would not be equality of thing, given and taken, i.e., of house and penknife.
    \end{quote}

    Here, Aquinas’s point (following Aristotle) is that \emph{some} form of proportional balancing is needed so that, for example, the combined labor or value put into \emph{many} sandals might properly match the labor or value of building a \emph{house}. Strict one-for-one exchange of different goods could be obviously unfair.

    \begin{quote}
        980. Next […], he shows how exchange takes place according to the preceding commensuration. Although a house is worth more than a sandal, nevertheless, a number of sandals are equal in value to one house or the food required for one man during a long period. [...] If this is not observed, there will be no exchange of things and men will not share their goods with one another.
    \end{quote}
    
    \begin{quote}
        983. Then […], he shows how just reciprocation takes place in exchanges according to the preceding commensuration. [...] He says first that the norm measuring all things by need according to nature and by currency according to human convention will then become reciprocation when everything will be equated in the way just mentioned.
    \end{quote}
    
    Aquinas explains Aristotle’s use of a “proportional figure with diagonals”—the well-known schematic in Book V of the \textit{Nicomachean Ethics}—to illustrate how different trades and values must be matched in some ratio. If the builder invests more labor or expense, that difference should be compensated by a correspondingly larger quantity of goods from the other party.

    \begin{quote}
        Thus, when exchange of things (\textit{commutatio rerum}) takes place, the articles to be exchanged (\textit{res commutandas}) ought to be arranged in a proportional figure with diagonals, as was stated previously (957). If this was not done, one extreme would have both excesses (\textit{superabundantias});
    
        if a farmer gave a bushel of wheat for a sandal, he would have an excess of labour in his product (\textit{superabundantiam laboris in opere}) and would have also an excess of presents (\textit{superabundantiam doni}) because he would be giving more than he would receive. But when all have what is theirs, they are in this way equal and do business with one another because the equality previously mentioned is possible for them.
    \end{quote}

\section*{Additional Context and Observations}

    \begin{remark}[Roman Law and the “Just Price”]
        Aquinas’s notion of \emph{just price} was influenced by Roman legal principles, such as those found in the \textit{Codex Iustinianus}. One doctrine, \textit{laesio enormis}, allowed a contract to be rescinded if it was made at less than half of the fair market value. Aquinas often reconciles such legal norms with moral theology: human law tolerates smaller deviations but punishes larger ones.
    \end{remark}

    \begin{remark}[Moral vs. Legal Dimensions]
        Aquinas’s distinction between the requirements of \emph{human law} and \emph{divine law} underscores that, from the standpoint of perfect virtue, even smaller injustices in exchange can be blameworthy. However, civil law ordinarily penalizes only the more severe cases so that social order remains manageable among “many lacking in virtue.”
    \end{remark}

    \begin{remark}[Natural vs. Unnatural Exchange (Aristotle’s Framework)]
        Following Aristotle, Aquinas differentiates between (a) \emph{natural exchange}, which serves to provide for genuine needs (e.g., household provisioning) and is seen as ethically commendable, and (b) \emph{unnatural or artificial exchange}, which solely pursues profit or speculation. This latter is neither intrinsically sinful nor virtuous but can become morally questionable when it stems from greed or exploits others.
    \end{remark}

    \begin{remark}[Labour, Risk, and Transportation]
        Aquinas consistently holds that a higher selling price may be justified when it compensates for improvements, labour, transport costs, or risk borne by the merchant. Thus, “buying low and selling high” is not automatically unjust; it depends on whether the markup is proportionate to real added value or costs.
    \end{remark}

    \begin{remark}[Practical Applicability]
        Although Aquinas lived in a medieval context (13th century), his arguments about honesty in trade, transparency of defects, and fairness in pricing remain influential in discussions of market ethics, consumer protection, and economic justice today.
    \end{remark}

        \subsubsection{Final considerations}

    \noindent In summary, Aquinas’s teaching on just price centres on the principle that transactions must preserve a certain equality, measured not only by the intrinsic worth of the item but also by factors such as labour input, risk undertaken, and legitimate need of both parties. While human law may choose to tolerate minor deviations from perfect equity, moral theology requires that all intentional distortion or fraud be avoided. Exchanges should, as far as possible, reflect a mutual benefit and fair equivalence between what is given and what is received—a principle that finds its roots in both Aristotelian philosophy and the broader Christian moral tradition.
